\documentclass[10pt]{beamer}
\usepackage[utf8]{inputenc}
\usepackage[T1,T2A]{fontenc}
\usepackage[russian]{babel}
\usepackage{color}
\usepackage{calc}
\usepackage{graphicx}
\usepackage{epstopdf}
\usepackage{hyperref}
\hypersetup{unicode,colorlinks}
\usepackage{csquotes}
\usepackage{upquote}
\usepackage{cprotect}
\usetheme[progressbar=head,numbering=fraction,block=fill]{metropolis}
\usepackage{dejavu}
\usepackage{etoolbox}
\usepackage{bm}
\usepackage[ND]{prftree}
\usepackage[tableaux]{prooftrees}
\usepackage{mathtools} % for bigtimes
\usepackage{diagbox}
\usepackage{verbatimbox}
\usepackage{fitch}
\usepackage{enumitem}
%\usepackage[defaultsans]{droidsans}
%\usepackage[defaultmono]{droidmono}
%\makeatletter
%\input droid/t1fdm.fd
%\makeatother

\makeatletter
\DeclareRobustCommand*{\bfseries}{%
    \not@math@alphabet\bfseries\mathbf
    \fontseries\bfdefault\selectfont
    \boldmath
}
\makeatother

\makeatletter
\chardef\straightquote@code=\catcode`'
\chardef\backquote@code=\catcode``
\catcode`'=\active \catcode``=\active
\patchcmd{\@noligs}
{\textasciigrave}
{\fixedtextasciigrave}
{}{}
\newcommand{\fixedtextasciigrave}{%
    \makebox[.5em]{\fontencoding{TS1}\fontfamily{fvs}\selectfont\textasciigrave}% Vera Sans
}
\catcode\lq\'=\straightquote@code
\catcode\lq\`=\backquote@code
\makeatletter

\setbeamercovered{again covered={\opaqueness<1->{25}}}

\forestset{close with=\ensuremath{\bigtimes}}

\newcommand{\customframefont}[1]{
    \setbeamertemplate{itemize/enumerate body begin}{#1}
    \setbeamertemplate{itemize/enumerate subbody begin}{#1}
}

\NewEnviron{framefont}[1]{
    \customframefont{#1} % for itemize/enumerate
    {#1 % For the text outside itemize/enumerate
        \BODY
    }
    \customframefont{\normalsize}
}


\makeatletter
\DeclareRobustCommand{\rvdots}{%
    \vphantom{\int\limits^x}\smash[t]{\vdots}
}
\makeatother

\newcommand{\N}{\mathbb{N}}
\newcommand{\Z}{\mathbb{Z}}
\newcommand{\Q}{\mathbb{Q}}
\newcommand{\R}{\mathbb{R}}
\newcommand{\pow}[1]{\mathcal{P}({#1})}
\newcommand{\continuum}{\mathfrak{c}}

\author{Алексей Романов}
\subtitle{Математическая логика и теория алгоритмов}
%\logo{}
\institute{МИЭТ}
\subject{Математическая логика и теория алгоритмов}
%\setbeamercovered{transparent}
%\setbeamertemplate{navigation symbols}{}


\title{Введение в логику предикатов}
\date{\today}

\begin{document}
\begin{frame}[plain]
\maketitle
\end{frame}

\begin{frame}
    \frametitle{Структура атомарных высказываний}
    \begin{itemize}
        \item Вспомним примеры простых высказываний с прошлой лекции: $4<5$, \enquote{Волга впадает в Балтийское море}.
        \pause
        \item $4$, $5$, \enquote{Волга} и \enquote{Балтийское море} здесь \emph{константы}, обозначающие какие-то объекты.
        \item $<$ и \enquote{\_ впадает в \_} "--- \emph{предикаты}, обозначающие отношения между объектами. Оба предиката \emph{бинарные} (или \emph{двухместные}).
        \pause
        \item Свойства (например \enquote{\_ круглое} в \enquote{Земля круглая}) "--- тоже предикаты, но \emph{унарные}.
        \pause
        \item Ещё пример: $2+2=5$. Здесь $2$ и $5$ "--- \pause константы, $=$ "--- \pause предикат, а $+$ "--- \pause \emph{функция}.
        \item $2+2$ "--- \emph{терм}, обозначает объект (как и константы).
        \item В $n$-местный предикат можно подставить $n$ термов и получить высказывание.
    \end{itemize}
\end{frame}

\begin{frame}
    \frametitle{Сигнатуры, термы, формулы}
    \begin{itemize}
        \item \emph{Сигнатура} $\Sigma$ "--- конечные или счётные множества константных $Const_\Sigma$, функциональных $Fun_\Sigma$ и предикатных $Pred_\Sigma$ символов. Для каждого символа из $Fun_\Sigma$ и $Pred_\Sigma$ задано число аргументов (\emph{арность}).
        \item \emph{Переменные} "--- \pause $x,\,y,\,z_3,\,\ldots$. Обозначают \pause какие-то объекты (не истину/ложь, как $p,q,r$). Множество $Var$ не зависит от сигнатуры. \pause
        \item \emph{Термы} "--- \pause $x,\,(y+2) \cdot z,\,\ldots$ Выражения, значения которых "--- объекты. Строятся из переменных и константных символов применением функциональных. Множество термов над $\Sigma$: $Term_\Sigma$. \pause
        \item \emph{Формулы} "--- \pause $x=y+1$; $\forall x ~ x \neq x^2$ \ldots Вот их значения истина и ложь. \emph{Атомарные формулы} строятся из термов применением предикатных символов, а остальные  применением связок $\land$/$\lor$/\ldots и кванторов $\forall$ и $\exists$ к формулам. Множество формул над $\Sigma$: $Form_\Sigma$.
    \end{itemize}
\end{frame}

\begin{frame}
    \frametitle{Модели}
    \begin{itemize}
        \item \emph{Модель} $M$ сигнатуры $\Sigma$ состоит из:
        \begin{itemize}
            \item \emph{Носитель} (или \emph{универсум}): \pause множество $\bar{M}$.
            \item Для каждого символа $c : Const_\Sigma$\only<2->{\footnote{Я пишу $:$ вместо $\in$ для объявления новой переменной.}}: \pause элемент $c_M : \bar{M}$.
            \item Для $f: Fun_\Sigma$: \pause функция $f_M: M^{arity(f)} \to M$.
            \item Для $P: Pred_\Sigma$: \pause предикат $P_M: M^{arity(P)} \to \{0,1\}$.
        \end{itemize}
        \item \emph{Оценка} "--- значение переменных в модели, $\sigma : Var \to \bar{M}$. Или $V \to \bar{M}$, где $V \subset Var$.
        \item Важно: символы сигнатуры сами по себе ничего не говорят об их интерпретации! $+_M$ может быть $-$ или $\cdot$ или $x,y \mapsto \int_{x}^{y} t dt$.
        \pause 
        \item Одно исключение: если есть $=$, то $=_M$ это равенство на $\bar{M}$.
    \end{itemize}
\end{frame}

\begin{frame}
    \frametitle{Значения термов и формул}
    \begin{itemize}
        \item $\sigma$ расширяется на \only<1>{$Term_\Sigma \to ~ ???$ и на $Form_\Sigma \to ~ ???$}\only<2->{$Term_\Sigma \to \bar{M}$ и на $Form_\Sigma \to \{0,1\}$} по индукции: \pause 
        \item $\sigma(c) = c_M$.
        \item $\sigma(f(t_1,\ldots,t_n)) = f_M(\sigma(t_1),\ldots,\sigma(t_n))$.
        \pause
        \item[]
        \item $\sigma(P(t_1,\ldots,t_n)) = P_M(\sigma(t_1),\ldots,\sigma(t_n))$.
        \item $\sigma(A \land B) = \sigma(A) \land \sigma(B)$ (аналогично для $\neg$/$\lor$/$\to$).
        \pause
        \item Для кванторов нужно сначала определить $\sigma_{v \mapsto a}$ (где $v$ переменная, а $a$ "--- объект из $\bar{M}$): 
        \[ \begin{array}{l}
            \sigma_{v \mapsto a}(v) = a \\
            \sigma_{v \mapsto a}(u) = \sigma(u) \text{ для остальных переменных } u.
        \end{array} \]
        \pause
        Теперь
        \[ \sigma(\forall v ~ A) = \left\{ \begin{array}{l}
            1, \text{ если } \forall a : \bar{M} ~  \sigma_{v \mapsto a}(A) = 1 \\
            0 \text{ иначе }
        \end{array} \right. \]
        и аналогично для $\exists$.
    \end{itemize}
\end{frame}

\begin{frame}
    \frametitle{Свободные и связанные переменные}
    \begin{itemize}
        \item \emph{Связанное вхождение переменной} "--- \pause $v$ под квантором по этой же переменной $\forall v/\exists v$.
        \item \emph{Свободное вхождение переменной} "--- \pause любое другое.
        \item Переменная свободна в формуле, если имеет какие-то свободные вхождения.
        \item \emph{Замкнутая формула} (или \emph{предложение}) "--- формула без свободных переменных.
        \item Почему это важно? Какая между ними разница? \pause 
        \item $\sigma(\forall/\exists v ~ \ldots)$ не зависит от $\sigma(v)$.
        \pause
        \item Отсюда можно доказать, что значение формулы $A$ зависит только от значений свободных переменных $A$, но не от связанных.
        \pause
        \item А значение замкнутой формулы? \pause Вообще не зависит от оценки, только от модели.
    \end{itemize}
\end{frame}

\begin{frame}
    \frametitle{Свободные и связанные переменные (2)}
    \begin{itemize}
        \item Связанную переменную можно переименовать, не изменив смысла и значения формулы.
        \item Вместо свободной переменной можно подставить произвольный терм, вместо связанной нельзя.
        \item Переменные с одним названием, связанные разными кванторами, это по сути разные переменные. Пример: $x = 3 \land (\exists x ~ x=0) \land \forall x ~ (x > 0 \to x^2 > 0)$.
        \item В математике переменные могут связываться не только кванторами: \pause
        \[ \lim_{x \to \ldots} \ldots \qquad \int_{\ldots}^{\ldots} \ldots dx \qquad \{x|\ldots\} \]
        все связывают $x$.
    \end{itemize}
\end{frame}

\begin{frame}
    \frametitle{Примеры}
    \begin{itemize}
        \item Рассмотрим пример.
        \item Формула (замкнутая): $\forall x ~ (P(x) \to Q(x))$.
        \item Возьмём произвольную модель из 3 элементов: $\bar{M} = \{0,1,2\}$.
        \item Как можно задать предикаты $P$ и $Q$? \pause
        \item[] \begin{tabular}{c | c c c}
            x & 0 & 1 & 2 \\ \hline
            P(x) & 0 & 0 & 1 \\
            Q(x) & 1 & 0 & 1
        \end{tabular}
        \item Истинна ли формула? \pause Да. \pause
        \item Попробуем другую формулу на той же модели:  $\exists x ~ (P(x) \land Q(x)) \land \exists x ~ \neg P(x)$. Истинна ли она? \pause Да. \pause
        \item Как насчёт незамкнутой формулы $P(x) \to \forall x ~ Q(x)$? \pause Истинна при $\sigma(x) = 0,~\sigma(x) = 1$, ложна при $\sigma(x) = 2$. \pause
        \item Для простоты можно писать $x = 0$ вместо $\sigma(x) = 0$. \pause
        \item Ничего не изменится, если $\bar{M} = \{a,b,c\}$, какие-то абстрактные объекты. 
    \end{itemize}
\end{frame}

\begin{frame}
    \frametitle{Примеры}
    \begin{itemize}
        \item Ещё пример с бинарным предикатом:
        \item Формула: $\forall x \exists y ~ (P(x,y) \to P(x,x))$.
        \item Возьмём $\bar{M} = \{a,b,c\}$.
        \item Какой таблицей задаётся бинарный предикат? \pause
        \item[] \begin{tabular}{c | c c c}
            \diagbox[height=1.5\line]{x}{y} & a & b & c \\ \hline
            a & 0 & 1 & 1 \\
            b & 0 & 1 & 0 \\
            c & 1 & 1 & 0
        \end{tabular}
        \item Истинна ли формула? \pause Да. \pause
        \item Слегка поменяем: $\forall x ~ ((\exists y ~ P(x,y)) \to P(x,x))$. Эта формула \pause ложна. \pause
        \item На бесконечных моделях работает так же, но мы не можем просто перечислить все значения переменных, чтобы найти значение формулы! 
        \pause 
        \item Вместо механического процесса приходится думать.
        \item На практике для больших конечных тоже.
    \end{itemize}
\end{frame}

\begin{frame}
   \frametitle{Формализация}
   \begin{itemize}
       \item Часто возникает задача: дано утверждение (математическое или в терминах \enquote{реального мира}), нужно записать его в виде формулы данной сигнатуры.
       \item Пример: \enquote{Кто-то любит всех на свете}. Универсум: люди, предикат $Loves(x, y)$ для \enquote{$x$ любит $y$}.
       \item Кто-то любит всех на свете $\equiv \exists x $ $x$ любит всех на свете \pause $\equiv \exists x ~ \forall y $ $x$ любит $y$ \pause $\equiv \exists x ~ \forall y ~ Loves(x, y)$. \pause
    %    \item Ещё пример в той же сигнатуре: \enquote{Всякая любовь взаимна}.
    %    \item Ответ: \pause $\forall x ~ \forall y ~ (Loves(x, y) \to Loves(y,x))$. \pause
    %    \item И ещё: \enquote{Кто-то не любит никого, кто любит его}.
    %    \item Ответ: \pause $\exists x ~ \forall y ~ (Loves(y, x) \to \neg Loves(x,y))$.
   \end{itemize}
\end{frame}

\begin{frame}
    \frametitle{Ещё примеры формализации}
    \begin{itemize}
        \item \enquote{$x$ делится на $2$}. Универсум: натуральные числа.
        \item Если предположили что-то вроде $x / 2 \in \mathbb{N}$: это не подойдёт. Почему? \pause
        \item Например, $\mathbb{N}$ это не объект нашего универсума. А если $\in \mathbb{N}$ рассматривать как единый предикатный символ, он верен для всех объектов! \pause 
        \item Более того, если $/$ "--- функциональный символ, то он должен иметь значение в нашем универсуме для любых аргументов. Есть варианты логики, которые снимают это ограничение, но мы их не изучаем. \pause
        \item Ответ (возможный): \pause $\exists y ~ x = 2 \cdot y$. \pause
        \item Можно ли то же самое записать без $\cdot$? \pause Да! $\exists y ~ x = y + y$. \pause
        \item Вот \enquote{$x$ делится на $y$} без $\cdot$ записать уже не получится. \pause
     %    \item Задание сложнее: \enquote{$x$ "--- простое число}.
     %    \item Ответ (возможный): \pause $\forall y ~ \forall z ~ x = y \cdot z \to y = 1 \lor z = 1$. 
     %    \pause 
     %    \item Неправда! В чём ошибка? \pause $x \neq 1 \land \forall y ~ \forall z ~ x = y \cdot z \to y = 1 \lor z = 1$. 
    \end{itemize}
 \end{frame}
 
\begin{frame}
   \frametitle{Формализация со свободными переменными}
   \begin{itemize}
       \item У нас на предыдущих слайдах появлялись утверждения с переменными (например, \enquote{$x$ любит всех на свете}) в промежуточных результатах. 
       \item Может и сразу быть дано такое утверждение. \pause
       \item В результате должна получиться формула с теми же \textbf<3->{свободными} переменными (и какими угодно связанными). \pause
       \item Если в формуле есть \enquote{лишние} свободные переменные или связана одна из тех, что есть в формализуемом утверждении, это заведомо неверный ответ.
   \end{itemize}
\end{frame}

% \begin{frame}
%    \frametitle{Ограниченные кванторы}
%    \begin{itemize}
%        \item В математических текстах мы часто видим что-то вроде $\forall x > 1 ~ x^2 > x$. Но по нашему определению это не формула! В чём дело? \pause
%        \item Это сокращённая запись формулы, но какой?
%        \item $\forall x ~ (x > 1 ~ ? ~ x^2 > x)$. Какую связку нужно поставить?
%        \pause
%        \item $\forall x ~ (x > 1 \to x^2 > x)$. 
%        \pause
%        \item А для случая $\exists x > 1 ~ x^2 > x$?
%        \pause
%        \item $\exists x ~ (x > 1 \land x^2 > x)$.
%        \pause
%        \item Убедитесь, что это работает, если заменить $x > 1$ и $x^2 > 1$ на любые другие \pause формулы.
%    \end{itemize}
% \end{frame}

% \begin{frame}
%    \frametitle{\only<1>{$\exists !$}\only<2->{Существование и единственность}}
%    \begin{itemize}
%        \item $\exists ! x ~ P(x)$ читается как \pause \enquote{Существует единственное $x$ такое, что\ldots}
%        \item $P$ здесь "--- формула со свободной переменной $x$. \pause
%        \item Но в нашем языке такого символа нет.
%        \item Может быть, $!$ "--- предикатный символ (или функциональный, или константный)? \pause Тогда бы не получилась формула. После квантора может стоять только переменная. \pause
%        \item Это сокращение, как и ограниченные кванторы. Осталось его расшифровать. \pause
%        \item $\exists x ~ (P(x) ~\land~ ???)$ \pause
%        \item $\exists x ~ (P(x) \land \forall y ~ (P(y) \to x = y))$ или 
%        \item $(\exists x ~ P(x)) \land \forall y,z ~ (P(y) \land P(z) \to y = z)$
%    \end{itemize}
% \end{frame}

\begin{frame}
    \frametitle{Многосортная логика предикатов}
    \begin{itemize}
        \item Часто удобно одновременно говорить о нескольких разных типах объектов. Пример: числа, множества чисел и функции в мат. анализе. Тогда
        \item К сигнатуре добавляется набор \emph{сортов}. Каждый сорт обозначает какое-то множество объектов.
        \item У функциональных и предикатных символов кроме числа аргументов задан сорт каждого, у функциональных ещё и сорт результата. 
        \item Применение символов к аргументам не тех сортов считается бессмысленным (т.е. его результат не является термом/формулой). 
        \item Каждая переменная имеет сорт: $x:S$. Сорт термов определяется по индукции.
        \item В моделях есть носитель для каждого сорта.
        \item Многосортную логику можно свести к односортной, добавив по предикату для каждого сорта, но формулы при этом усложняются.
    \end{itemize}
\end{frame}

\begin{frame}
    \frametitle{Подстановки}
    \begin{itemize}
        \item Попробуем определить результат замены переменной $v$ на терм $t$ в терм $s$ и в формулу $A$: $s[v \mapsto t]$ и $A[v \mapsto t]$. Должно получиться 
        \[ \begin{array}{c}
            \sigma(s[v \mapsto t]) = \sigma_{v \mapsto \sigma(t)}(s) \\
            \sigma(A[v \mapsto t]) = \sigma_{v \mapsto \sigma(t)}(A)
        \end{array} \]
        (понятен ли смысл этого?). 
        \pause
        \item Первые шаги очевидны:
        \[ \begin{array}{l}
            v[v \mapsto t] = t \\ \pause
            u[v \mapsto t] = \pause u ~(u\text{ "--- переменная, кроме }v) \\
            c[v \mapsto t] = \pause c ~(c \text{ "--- константа}) \\ \pause
            f(t_1,\ldots,t_n)[v \mapsto t] = \pause f(t_1[v \mapsto t],\ldots,t_n[v \mapsto t]) \\ \pause
            P(t_1,\ldots,t_n)[v \mapsto t] = \pause P(t_1[v \mapsto t],\ldots,t_n[v \mapsto t]) \\ \pause
            (\neg A)[v \mapsto t] = \neg (A[v \mapsto t]) ~ (\text{аналогично для } \land/\lor/\to)
        \end{array} \]
    \end{itemize}
\end{frame}

\begin{frame}
    \frametitle{Подстановки с кванторами}
    \begin{itemize}
        \item Для формул с кванторами всё сложнее.
        \item Если квантор по той же переменной:
        \[ (\forall v ~ A)[v \mapsto t] = \pause \forall v ~ A \]
        Формула не меняется!
        \pause
        \item Если переменные разные, можно предположить 
        \[ (\forall u ~ A)[v \mapsto t] = \pause \forall u ~ (A[v \mapsto t]) \]
        \pause
        \vspace{-1.5em}
        \item Но рассмотрим пример
        \enquote{$x$ делится на $2$} $\equiv \exists y ~ x = 2 \cdot y$. При замене $x$ на $y$ должно получиться \pause \enquote{$y$ делится на $2$}. Если попробуем правило выше, получится \pause $\exists y ~ y = 2 \cdot y$, \pause явно не выражающая \enquote{$y$ делится на $2$}.
        \item Если в $A$ связаны какие-то переменные терма $t$, то перед подстановкой их нужно сначала переименовать: 
        \[ (\exists y ~ x = 2 \cdot y)[x \mapsto y] = (\exists y' ~ x = 2 \cdot y')[x \mapsto y] = \exists y' ~ y = 2 \cdot y' \]
    \end{itemize}
\end{frame}

\end{document}