\documentclass[10pt]{beamer}
\usepackage[utf8]{inputenc}
\usepackage[T1,T2A]{fontenc}
\usepackage[russian]{babel}
\usepackage{color}
\usepackage{calc}
\usepackage{graphicx}
\usepackage{epstopdf}
\usepackage{hyperref}
\hypersetup{unicode,colorlinks}
\usepackage{csquotes}
\usepackage{upquote}
\usepackage{cprotect}
\usetheme[progressbar=head,numbering=fraction,block=fill]{metropolis}
\usepackage{dejavu}
\usepackage{etoolbox}
\usepackage{bm}
\usepackage[ND]{prftree}
\usepackage[tableaux]{prooftrees}
\usepackage{mathtools} % for bigtimes
\usepackage{diagbox}
\usepackage{verbatimbox}
\usepackage{fitch}
\usepackage{enumitem}
%\usepackage[defaultsans]{droidsans}
%\usepackage[defaultmono]{droidmono}
%\makeatletter
%\input droid/t1fdm.fd
%\makeatother

\makeatletter
\DeclareRobustCommand*{\bfseries}{%
    \not@math@alphabet\bfseries\mathbf
    \fontseries\bfdefault\selectfont
    \boldmath
}
\makeatother

\makeatletter
\chardef\straightquote@code=\catcode`'
\chardef\backquote@code=\catcode``
\catcode`'=\active \catcode``=\active
\patchcmd{\@noligs}
{\textasciigrave}
{\fixedtextasciigrave}
{}{}
\newcommand{\fixedtextasciigrave}{%
    \makebox[.5em]{\fontencoding{TS1}\fontfamily{fvs}\selectfont\textasciigrave}% Vera Sans
}
\catcode\lq\'=\straightquote@code
\catcode\lq\`=\backquote@code
\makeatletter

\setbeamercovered{again covered={\opaqueness<1->{25}}}

\forestset{close with=\ensuremath{\bigtimes}}

\newcommand{\customframefont}[1]{
    \setbeamertemplate{itemize/enumerate body begin}{#1}
    \setbeamertemplate{itemize/enumerate subbody begin}{#1}
}

\NewEnviron{framefont}[1]{
    \customframefont{#1} % for itemize/enumerate
    {#1 % For the text outside itemize/enumerate
        \BODY
    }
    \customframefont{\normalsize}
}


\makeatletter
\DeclareRobustCommand{\rvdots}{%
    \vphantom{\int\limits^x}\smash[t]{\vdots}
}
\makeatother

\newcommand{\N}{\mathbb{N}}
\newcommand{\Z}{\mathbb{Z}}
\newcommand{\Q}{\mathbb{Q}}
\newcommand{\R}{\mathbb{R}}
\newcommand{\pow}[1]{\mathcal{P}({#1})}
\newcommand{\continuum}{\mathfrak{c}}

\author{Алексей Романов}
\subtitle{Математическая логика и теория алгоритмов}
%\logo{}
\institute{МИЭТ}
\subject{Математическая логика и теория алгоритмов}
%\setbeamercovered{transparent}
%\setbeamertemplate{navigation symbols}{}


\title{Натуральная дедукция (естественный вывод) для логики предикатов}
\date{\today}

\begin{document}
\begin{frame}[plain]
    \maketitle
\end{frame}

\begin{frame}
    \frametitle{Правила для логики высказываний}
    \scriptsize
    \begin{itemize}
        \item Для $\land$:
        \[ 
        \frac{A \qquad B}{A \wedge B}\ \wedge I 
        \qquad \qquad 
        \frac{A \wedge B}{A}\ \wedge{E}
        \qquad
        \frac{A \wedge B}{B}\ \wedge{E} 
        \]
        \item Для $\to$:
        \[ 
        \frac{\begin{matrix}
                A \\
                \rvdots \\
                B
        \end{matrix}}{A \to B}\ \to I 
        \qquad \qquad 
        \frac{A \to B \quad A}{B}\ \to E
        \]
        \item Для $\neg$ и $\bot$:
        \[
        \frac{\begin{matrix}
                A \\
                \rvdots \\
                \bot
        \end{matrix}}{\neg A}\ \neg I 
        \qquad \qquad 
        \frac{\neg A \quad A}{\bot}\ \neg E/\bot I
        \qquad \qquad 
        \frac{\bot}{A}\ \bot{E}
        \]
        \item Для $\lor$:
        \[
        \frac{A}{A \lor B}\ \lor{I}
        \qquad
        \frac{B}{A \lor B}\ \lor{I}
        \qquad \qquad 
        \frac{
        \begin{matrix}
            \phantom{A} \\
            \phantom{\rvdots} \\
            A \lor B 
        \end{matrix}
        \quad
        \begin{matrix}
            A \\
            \rvdots \\
            C
        \end{matrix}
        \quad
        \begin{matrix}
            B \\
            \rvdots \\
            C
        \end{matrix}
        }{C}\ \lor E
        \qquad \qquad         
        \frac{\begin{matrix}
            \neg A \\
            \rvdots \\
            B
\end{matrix}}{A \lor B}\ \lor I'
        \qquad 
\frac{\begin{matrix}
        \neg B \\
        \rvdots \\
        A
\end{matrix}}{A \lor B}\ \lor I'
        \]
        \item Остальные:
        \[
        \frac{\begin{matrix}
                \neg A \\
                \rvdots \\
                \bot
        \end{matrix}}{A}\ RAA
        \qquad \qquad 
        \frac{A}{A}\ R
        \]
    \end{itemize}
\end{frame}

\begin{frame}
    \frametitle{Правила натуральной дедукции для кванторов}
    \begin{itemize}
        \item Снова сохраняются правила из логики высказываний (на предыдущем слайде).
        \item Добавляются правила введения и исключения для кванторов. 
        \pause
        \item Начнём с простых: 
        $\dfrac{\forall x ~ A(x)}{A(t)}\ \forall E$ и
        $\dfrac{A(t)}{\exists x ~ A(x)}\ \exists I$. \\
        Здесь $t$ --- любой терм без свободных переменных (обычно просто параметр).
        \item Эти правила --- аналоги типа $\gamma$ в деревьях.
        \pause
        \item
        $\dfrac{\boxed{
            \begin{matrix}
                \textrm{произв. }a \\
                \rvdots \\
                A(a)
            \end{matrix}}}{\forall x ~ A(x)}\ \forall I$ и  
        $\dfrac{
            \begin{matrix}
                \phantom{a} \\
                \phantom{A} \\
                \phantom{\rvdots} \\
                \exists x ~ A(x)
            \end{matrix}
            \quad
            \boxed{
                \begin{matrix}
                    \textrm{возьмём }a, \\
                    \textrm{т.ч. }A(a) \\
                    \rvdots \\
                    B
                \end{matrix}
            }
         }{B}\ \exists E$
          \\
        Здесь $a$ --- новый параметр, как в $\delta$.
    \end{itemize}
\end{frame}

\begin{frame}
    \frametitle{Пояснения к правилам}
    \begin{itemize}
        \item Смысл правила $\forall I$: чтобы доказать $\forall x ~ A(x)$, нужно доказать $A(a)$ для \emph{произвольного} $a$. Произвольность как раз обеспечивается тем, что это новый параметр и о нём ничего неизвестно. \pause
        \item Смысл правила $\exists E$: мы временно даём название $a$ тому объекту, существование которого утверждается. Правило немного аналогично $\lor E$.
        \item Удобно помечать подвывод, вводящий параметр, этим параметром.
        \item[]
        \item Есть упрощённый вариант $\exists E$, не вводящий подвывод, но он усложняет определение контрмоделей для недоказумых секвенций и мы его не используем.
    \end{itemize}
\end{frame}


\begin{frame}
    \frametitle{Пример 1}
    \begin{align*}
    \forall x (P(x) \rightarrow Q(x)) \vdash \exists x P(x) \rightarrow \exists x Q(x) \\
    \vspace{1em}
    \only<1>{
    \begin{nd}
        \hypo{1}{\forall x (P(x) \rightarrow Q(x))}  \by{\text{Дано}}{}
        \have[\vdots]{4}{\vdots}
        \have[n]{n}{\exists x P(x) \rightarrow \exists x Q(x)}
    \end{nd}
    }
    \only<2>{
    \begin{nd}
        \hypo{1}{\forall x (P(x) \rightarrow Q(x))}  \by{\text{Дано}}{}
        \open
        \hypo{3}{\exists x P(x)} \by{\text{Дано}}{}
        \have[\vdots]{7}{\vdots}
        \have[n][-1]{4}{\exists x Q(x)} 
        \close
        \have[n]{2}{\exists x P(x) \rightarrow \exists x Q(x)} \ii{3-(4)}
    \end{nd}
    }
    \only<3>{
    \begin{nd}
        \hypo{1}{\forall x (P(x) \rightarrow Q(x))}  \by{\text{Дано}}{}
        \open
        \hypo{3}{\exists x P(x)} \by{\text{Дано}}{}
        \open[a]
        \hypo{5}{P(a)} \by{\text{Дано}}{}
        \have[\vdots]{7}{\vdots}
        \have[n][-2]{6}{\exists x Q(x)}
        \close
        \have[n][-1]{4}{\exists x Q(x)} \Ee{3,5-(6)}
        \close
        \have[n]{2}{\exists x P(x) \rightarrow \exists x Q(x)} \ii{3-(4)}
    \end{nd}
}
    \only<4->{
    \begin{nd}
        \hypo{1}{\forall x (P(x) \rightarrow Q(x))}  \by{\text{Дано}}{}
        \open
        \hypo{3}{\exists x P(x)} \by{\text{Дано}}{}
        \open[a]
        \hypo{5}{P(a)} \by{\text{Дано}}{}
        \have{7}{P(a) \rightarrow Q(a)} \Ae{1}
        \have{8}{Q(a)} \ie{5,7}
        \have[n][-2]{6}{\exists x Q(x)}
        \close
        \have[n][-1]{4}{\exists x Q(x)} \Ee{3,5-(6)}
        \close
        \have[n]{2}{\exists x P(x) \rightarrow \exists x Q(x)} \ii{3-(4)}
    \end{nd}
    }
    \end{align*}
\end{frame}

\begin{frame}
    \frametitle{Пример 2}
    \begin{align*}
    \forall y \neg P(y) \vdash \neg \exists x P(x) \\
    \vspace{1em}
    \begin{nd}
        \hypo {1} {\forall y \neg P(y)} \by{\text{Дано}}{}
        \open
        \hypo{2} {\exists x P(x)} \by{\text{Дано}}{}
        \open[a]
        \hypo{3} {P(a)} \by{\text{Дано}}{}
        \have{5} {\neg P(a)}            \Ae{1}
        \have[n-2]{6} {\bot}                 \ne{3,5}
        \close
        \have[n-1]{6a}{\bot}                 \Ee{2,3-(6)}
        \close
        \have[n]{7} {\neg \exists x P(x)}  \ni{2-(6a)}
    \end{nd}
    \end{align*}
\end{frame}

\begin{frame}
    \frametitle{Пример 3}
    \begin{itemize}
        \item Докажем $\forall x ~ P(x) \land Q(x) \vdash (\forall x ~ P(x)) \land (\forall x ~ Q(x))$:
        \[
        \NDANDI
            {\prftree[r]{\scriptsize $\forall I\,a_1$}
                {\fbox{\NDANDE
                    {\NDALLE
                        {\prfbyaxiom{\scriptsize дано}{\forall x ~ P(x) \land Q(x)}}
                        {P(a_1) \land Q(a_1)}}
                    {P(a_1)}}}
                {\forall x ~ P(x)}}
            {\prftree[r]{\scriptsize $\forall I\,a_2$}
                {\fbox{\NDANDE
                        {\NDALLE
                            {\prfbyaxiom{\scriptsize дано}{\forall x ~ P(x) \land Q(x)}}
                            {P(a_2) \land Q(a_2)}}
                        {Q(a_2)}}}
                {\forall x ~ Q(x)}}
            {(\forall x ~ P(x)) \land (\forall x ~ Q(x))}
        \]
        \item Можно ли оба раза использовать $a_1$? \pause Да, так как подвыводы независимы. Но незачем.
    \end{itemize}
    \end{frame}

\begin{frame}
    \frametitle{Пример 3 в линейной записи}
        \begin{columns}
            \begin{column}{0.5\textwidth}
                    $\scriptsize
                \begin{nd}
                \hypo{1}{\forall x ~ P(x) \land Q(x)} \by{\text{Дано}}{}
                \open[a_1]
                \hypo{a_1}{\text{произв. }a_1}
                \have{2}{P(a_1) \land Q(a_1)} \Ae{1}
                \have[k][-1]{3}{P(a_1)} \ae{2}
                \close
                \open[a_2]
                \hypo{a_2}{\text{произв. }a_2}
                \have[k]{5}{P(a_2) \land Q(a_2)} \Ae{1}
                \have[n][-3]{6}{Q(a_2)} \ae{5}
                \close
                \have[n][-2]{4}{\forall x ~ P(x)} \Ai{a_1-(3)}
                \have[n][-1]{7}{\forall x ~ Q(x)} \Ai{a_2-(6)}
                \have[n]{8}{(\forall x ~ P(x)) \land (\forall x ~ Q(x))} \ai{4,7}
                \end{nd}
                $ \\ \vspace{0.5em} или
            \end{column} \pause
            \begin{column}{0.05\textwidth}\end{column}
            \begin{column}{0.5\textwidth}
        $\scriptsize
        \begin{nd}
        \hypo{1}{\forall x ~ P(x) \land Q(x)} \by{\text{Дано}}{}
        \have[][7]{2}{P(a_1) \land Q(a_1)} \Ae{1}
        \open[a_1]
        \hypo{a_1}{\text{произв. }a_1}
        \have[][5]{3}{P(a_1)} \ae{2}
        \close
        \open[a_1]
        \hypo{a_2}{\text{произв. }a_1}
        \have[][6]{6}{Q(a_1)} \ae{5}
        \close
        \have[][3]{4}{\forall x ~ P(x)} \Ai{a_1-3}
        \have{7}{\forall x ~ Q(x)} \Ai{a_2-6}
        \have[][2]{8}{(\forall x ~ P(x)) \land (\forall x ~ Q(x))} \ai{4,7}
        \end{nd}
        $ \\ \vspace{0.5em} Здесь важно, что строка 7 появилась после 5 и 6, иначе в них нельзя использовать $a_1$.
        \end{column}
    \end{columns}
\end{frame}

\begin{frame}
    \frametitle{Пример 4}
    \begin{itemize}
        \item Доказываем $\exists x \forall y ~ P(x, y) \vdash \forall y \exists x ~ P(x, y)$: \pause
        \[
        \prftree[r]{\scriptsize $\forall I\,a_1$}
        {\fbox{
                \prftree[r]{\scriptsize $\exists E\,a_2; \forall y ~ P(a_2, y) \vdash$}
                {\prfbyaxiom{\scriptsize дано}{\exists x \forall y ~ P(x, y)}}
                {\fbox{
                        \NDEXI
                            {\NDALLE{\prfbyaxiom{\scriptsize для $\exists E$}{\forall y ~ P(a_2, y)}}{P(a_2, a_1)}}
                            {\exists x ~ P(x, a_1)}
                }}
                {\exists x ~ P(x, a_1)}
        }}
        {\forall y \exists x ~ P(x, y)}
        \]
    \end{itemize}
\end{frame}

\begin{frame}
    \frametitle{Пример 4 в линейной записи}
    \begin{itemize}
        \item Доказываем $\exists x \forall y ~ P(x, y) \vdash \forall y \exists x ~ P(x, y)$ в линейной записи:
        \item[]
        \[
        \begin{nd}
        \hypo{1}{\exists x \forall y ~ P(x, y)} \by{\text{Дано}}{}
        \open[a_1]
        \hypo{a_1}{\text{произв. }a_1}
        \open[a_2]
        \hypo{4}{\forall y ~ P(a_2, y)} \by{\text{Дано}}{}
        \have{5}{P(a_2, a_1)} \Ae{4}
        \have[n][-2]{6}{\exists x ~ P(x, a_1)} \Ei{5}
        \close
        \have[n][-1]{3}{\exists x ~ P(x, a_1)} \Ee{1,4-(6)}
        \close
        \have[n]{2}{\forall y \exists x ~ P(x, y)} \Ai{a_1-(3)}
        \end{nd}
        \]
    \end{itemize}
\end{frame}

\begin{frame}
    \frametitle{Построение контрмодели}
    \begin{itemize}
        \item Если формулу или секвенцию доказать не получается, можно предположить, что она неверна и попробовать построить контрмодель.
        \item Для дерева с вынесенными посылками выбираем вершину, а для линейной записи --- строку, на которых застряли. 
        \item Все видимые посылки должны быть истинны, а сама выбранная формула ложна.
        \item Значения предикатов должны быть заданы на всех параметрах этих формул (если они все без параметров, то на одном параметре $a_1$). 
        \item Может понадобиться добавить ещё параметры.
    \end{itemize}
\end{frame}

\begin{frame}
    \frametitle{Функциональные символы}
    \begin{itemize}
        \item Пусть в сигнатуре есть функциональные символы или константы
        \item Тогда в правилах $\forall E$ и $\exists I$ можно использовать не только параметры, а произвольные термы, построенные из них: \pause $f(a_1)$, $a_3 + 0$ и так далее.
        \item Но без свободных переменных $x$, $y$, $\ldots$
        \item Простой пример: докажем $\forall x~P(x) \vdash \forall x~P(f(x))$: \pause
        \item[]
        \[
        \begin{nd}
        \hypo{1}{\forall x~P(x)} \by{\text{Дано}}{}
        \open[a]
        \hypo{a}{\text{произв. }a}
        \have[n][-1]{p}{P(f(a))} \Ae{1}
        \close
        \have[n]{n}{\forall x~P(f(x))} \Ai{a-(p)}
        \end{nd}
        \]
    \end{itemize}
\end{frame}

\begin{frame}
    \frametitle{Равенство}
    \begin{itemize}
        \item Для равенства вводятся новые правила введения и исключения: \pause
        \item $\dfrac{}{t = t}\ =I \qquad \dfrac{t = s \quad A[t]}{A[s]}\ =E \qquad \dfrac{s = t \quad A[t]}{A[s]}\ =E$
        \item $t$ и $s$ опять же термы без свободных переменных. 
        \item $A[t]$ "--- формула, содержащая терм $t$, $A[s]$ \pause "--- та же формула с заменой $t$ на $s$.
        \item Не обязательно заменять все вхождения $t$.
        \item Вместо правил можно было бы ввести аксиомы: $\forall x~x=x$ и $\forall x \forall y~(x=y \land A[x] \to A[y])$.
    \end{itemize}
\end{frame}

\begin{frame}
    \frametitle{Пример доказательства с равенством}
    \begin{itemize}
        \item Докажем симметричность равенства $\forall x \forall y~(x=y \to y=x)$: \pause
        \[
        \begin{nd}
        \open[\text{\scriptsize $a,b$}]
        \hypo{0}{\text{произв. }a,b}
        \open
        \hypo{1}{a = b} \by{\text{Дано}}{}
        \have{2}{b = b} \by{=I}{}
        \have[n][-2]{pp}{b = a} \by{=E\text{\small$(s,t\!: b, A[t]\!: b = \underline{b})$}}{1,2}
        \close
        \have[n][-1]{p}{a = b \to b = a} \ie{1-(pp)}
        \close
        \have[n]{n}{\forall x \forall y~(x=y \to y=x)} \Ai{0-(p)}
        \end{nd}
        \]
        \item Подчёркиванием в $=E$ выделено заменяемое вхождение терма.
    \end{itemize}
\end{frame}

\begin{frame}
    \frametitle{Дополнительное чтение}
    \begin{itemize}
        \item Непейвода, 11.2.5 и 11.4.
        \item Гладкий, глава 10 (менее удобная система записи).
    \end{itemize}
\end{frame}

\end{document}
