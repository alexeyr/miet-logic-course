\documentclass[10pt]{beamer}
\usepackage[utf8]{inputenc}
\usepackage[T1,T2A]{fontenc}
\usepackage[russian]{babel}
\usepackage{color}
\usepackage{calc}
\usepackage{graphicx}
\usepackage{epstopdf}
\usepackage{hyperref}
\hypersetup{unicode,colorlinks}
\usepackage{csquotes}
\usepackage{upquote}
\usepackage{cprotect}
\usetheme[progressbar=head,numbering=fraction,block=fill]{metropolis}
\usepackage{dejavu}
\usepackage{etoolbox}
\usepackage{bm}
\usepackage[ND]{prftree}
\usepackage[tableaux]{prooftrees}
\usepackage{mathtools} % for bigtimes
\usepackage{diagbox}
\usepackage{verbatimbox}
\usepackage{fitch}
\usepackage{enumitem}
%\usepackage[defaultsans]{droidsans}
%\usepackage[defaultmono]{droidmono}
%\makeatletter
%\input droid/t1fdm.fd
%\makeatother

\makeatletter
\DeclareRobustCommand*{\bfseries}{%
    \not@math@alphabet\bfseries\mathbf
    \fontseries\bfdefault\selectfont
    \boldmath
}
\makeatother

\makeatletter
\chardef\straightquote@code=\catcode`'
\chardef\backquote@code=\catcode``
\catcode`'=\active \catcode``=\active
\patchcmd{\@noligs}
{\textasciigrave}
{\fixedtextasciigrave}
{}{}
\newcommand{\fixedtextasciigrave}{%
    \makebox[.5em]{\fontencoding{TS1}\fontfamily{fvs}\selectfont\textasciigrave}% Vera Sans
}
\catcode\lq\'=\straightquote@code
\catcode\lq\`=\backquote@code
\makeatletter

\setbeamercovered{again covered={\opaqueness<1->{25}}}

\forestset{close with=\ensuremath{\bigtimes}}

\newcommand{\customframefont}[1]{
    \setbeamertemplate{itemize/enumerate body begin}{#1}
    \setbeamertemplate{itemize/enumerate subbody begin}{#1}
}

\NewEnviron{framefont}[1]{
    \customframefont{#1} % for itemize/enumerate
    {#1 % For the text outside itemize/enumerate
        \BODY
    }
    \customframefont{\normalsize}
}


\makeatletter
\DeclareRobustCommand{\rvdots}{%
    \vphantom{\int\limits^x}\smash[t]{\vdots}
}
\makeatother

\newcommand{\N}{\mathbb{N}}
\newcommand{\Z}{\mathbb{Z}}
\newcommand{\Q}{\mathbb{Q}}
\newcommand{\R}{\mathbb{R}}
\newcommand{\pow}[1]{\mathcal{P}({#1})}
\newcommand{\continuum}{\mathfrak{c}}

\author{Алексей Романов}
\subtitle{Математическая логика и теория алгоритмов}
%\logo{}
\institute{МИЭТ}
\subject{Математическая логика и теория алгоритмов}
%\setbeamercovered{transparent}
%\setbeamertemplate{navigation symbols}{}


\title{Натуральная дедукция (естественный вывод)}
\date{\today}

\begin{document}
\begin{frame}[plain]
    \maketitle
\end{frame}

\begin{frame}
    \frametitle{Правила натуральной дедукции}
    \scriptsize
    \begin{itemize}
        \item Для $\land$:
        \[ 
        \frac{A \qquad B}{A \wedge B}\ \wedge I 
        \qquad \qquad 
        \frac{A \wedge B}{A}\ \wedge{E}
        \qquad
        \frac{A \wedge B}{B}\ \wedge{E} 
        \]
        \item Для $\to$:
        \[ 
        \frac{\begin{matrix}
                A \\
                \rvdots \\
                B
        \end{matrix}}{A \to B}\ \to I 
        \qquad \qquad 
        \frac{A \to B \quad A}{B}\ \to E
        \]
        \item Для $\neg$ и $\bot$:
        \[
        \frac{\begin{matrix}
                A \\
                \rvdots \\
                \bot
        \end{matrix}}{\neg A}\ \neg I 
        \qquad \qquad 
        \frac{\neg A \quad A}{\bot}\ \neg E/\bot I
        \qquad \qquad 
        \frac{\bot}{A}\ \bot{E}
        \]
        \item Для $\lor$:
        \[
        \frac{A}{A \lor B}\ \lor{I}
        \qquad
        \frac{B}{A \lor B}\ \lor{I}
        \qquad \qquad 
        \frac{
        \begin{matrix}
            \phantom{A} \\
            \phantom{\rvdots} \\
            A \lor B 
        \end{matrix}
        \quad
        \begin{matrix}
            A \\
            \rvdots \\
            C
        \end{matrix}
        \quad
        \begin{matrix}
            B \\
            \rvdots \\
            C
        \end{matrix}
        }{C}\ \lor E
        \qquad \qquad         
        \frac{\begin{matrix}
            \neg A \\
            \rvdots \\
            B
\end{matrix}}{A \lor B}\ \lor I'
        \qquad 
\frac{\begin{matrix}
        \neg B \\
        \rvdots \\
        A
\end{matrix}}{A \lor B}\ \lor I'
        \]
        \item Остальные:
        \[
        \frac{\begin{matrix}
                \neg A \\
                \rvdots \\
                \bot
        \end{matrix}}{A}\ RAA
        \qquad \qquad 
        \frac{A}{A}\ R
        \]
    \end{itemize}
\end{frame}

\begin{frame}
    \frametitle{Правила натуральной дедукции для кванторов}
    \begin{itemize}
        \item Для $\forall$:
        \[ 
        \frac{\begin{matrix}
                \rvdots \\
                A(a)
        \end{matrix}}{\forall x ~ A(x)}\ \forall I 
        \qquad \qquad 
        \frac{\forall x ~ A(x)}{A(t)}\ \forall E
        \]
        \item Для $\exists$:
        \[
        \frac{A(t)}{\exists x ~ A(x)}\ \exists I
        \qquad \qquad 
        \frac{
            \begin{matrix}
                \phantom{A} \\
                \phantom{\rvdots} \\
                \exists x ~ A(x)
            \end{matrix}
            \quad
            \begin{matrix}
                A(a) \\
                \rvdots \\
                B
            \end{matrix}
        }{B}\ \exists E
        \]
        \item $t$ в $\forall E$ и $\exists I$ "--- произвольный терм (в наших примерах всегда просто параметр). Они соответствуют типу $\gamma$ в деревьях истинности.
        \item $a$ в $\forall I$ и $\exists E$ "--- новый параметр, которого нет в уже построенной части доказательства. Они соответствуют типу $\delta$.
    \end{itemize}
\end{frame}

\begin{frame}
    \frametitle{Пример}
    \begin{align*}
    \forall x (P(x) \rightarrow Q(x)) \vdash \exists x P(x) \rightarrow \exists x Q(x) \\
    \vspace{1em}
    \only<1>{
    \begin{nd}
        \hypo{1}{\forall x (P(x) \rightarrow Q(x))}  \by{\text{Дано}}{}
        \have[\vdots]{4}{\vdots}
        \have[][2]{2}{\exists x P(x) \rightarrow \exists x Q(x)}
    \end{nd}
    }
    \only<2>{
    \begin{nd}
        \hypo{1}{\forall x (P(x) \rightarrow Q(x))}  \by{\text{Дано}}{}
        \open
        \hypo[][3]{3}{\exists x P(x)} \by{\text{Дано}}{}
        \have[\vdots]{7}{\vdots}
        \have[][4]{4}{\exists x Q(x)} 
        \close
        \have[][2]{2}{\exists x P(x) \rightarrow \exists x Q(x)} \ii{3-4}
    \end{nd}
    }
    \only<3>{
    \begin{nd}
        \hypo{1}{\forall x (P(x) \rightarrow Q(x))}  \by{\text{Дано}}{}
        \open
        \hypo[][3]{3}{\exists x P(x)} \by{\text{Дано}}{}
        \open[a]
        \hypo[][5]{5}{P(a)} \by{\text{Дано}}{}
        \have[\vdots]{7}{\vdots}
        \have[][6]{6}{\exists x Q(x)}
        \close
        \have[][4]{4}{\exists x Q(x)} \Ee{3,5-6}
        \close
        \have[][2]{2}{\exists x P(x) \rightarrow \exists x Q(x)} \ii{3-4}
    \end{nd}
}
    \only<4->{
    \begin{nd}
        \hypo{1}{\forall x (P(x) \rightarrow Q(x))}  \by{\text{Дано}}{}
        \open
        \hypo[][3]{3}{\exists x P(x)} \by{\text{Дано}}{}
        \open[a]
        \hypo[][5]{5}{P(a)} \by{\text{Дано}}{}
        \have[][7]{7}{P(a) \rightarrow Q(a)} \Ae{1}
        \have{8}{Q(a)} \ie{5,7}
        \have[][6]{6}{\exists x Q(x)} \Ei{8}
        \close
        \have[][4]{4}{\exists x Q(x)} \Ee{3,5-6}
        \close
        \have[][2]{2}{\exists x P(x) \rightarrow \exists x Q(x)} \ii{3-4}
    \end{nd}
    }
    \end{align*}
\end{frame}

\begin{frame}
    \frametitle{Пример 2}
    \begin{align*}
    \forall y \neg P(y) \vdash \neg \exists x P(x) \\
    \vspace{1em}
    \begin{nd}
        \hypo {1} {\forall y \neg P(y)}
        \open
        \hypo[][3] {2} {\exists x P(x)}
        \open[a]
        \hypo[][5] {3} {P(a)}
        \have {5} {\neg P(a)}            \Ae{1}
        \have {6} {\bot}                 \ne{3,5}
        \close
        \have[][4] {6a}{\bot}                 \Ee{2,3-6}
        \close
        \have[][2]{7} {\neg \exists x P(x)}  \ni{2-6a}
    \end{nd}
    \end{align*}
\end{frame}

\end{document}