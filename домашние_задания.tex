\documentclass{article}

\usepackage[utf8]{inputenc}
\usepackage[T1,T2A]{fontenc}
\usepackage[russian]{babel}
\usepackage[a4paper, margin=1in]{geometry}
\usepackage{csquotes}
\usepackage{amsfonts}
\usepackage{amsmath}
\usepackage{amssymb}
\usepackage{hyperref}

\newcommand{\N}{\mathbb{N}}
\newcommand{\Z}{\mathbb{Z}}
\newcommand{\Q}{\mathbb{Q}}
\newcommand{\R}{\mathbb{R}}
\newcommand{\pow}[1]{\mathcal{P}({#1})}
\newcommand{\continuum}{\mathfrak{c}}

\newcommand{\itemhard}{\refstepcounter{enumi}\item[\theenumi$^\star$.]}
\renewcommand{\labelenumii}{\asbuk{enumii})}

\begin{document}
\section{Деревья истинности для логики высказываний}
Доказать или опровергнуть с помощью деревьев истинности:
\begin{enumerate}
    \item $((p \to q) \to p) \to p$
    \item $p \land q, q \land r \vdash p \land r$
    \item $p \lor q, q \lor r \vdash p \lor r$
    \item $p \to q \equiv \neg p \lor q$
    \item $\neg (p \to q) \equiv \neg p \to \neg q$
    \item $\neg (p \lor q) \equiv \neg p \land \neg q$
    \item $(p \land q) \lor (p \land \neg q) \lor (\neg p \land q) \lor (\neg p \land \neg q)$
    \item $p \land q \to r \equiv (p \to r) \lor (q \to r)$
    \item $p \land q \to r \equiv (p \to r) \land (q \to r)$
    \item \label{rules} Для каждой операции $op$ из следующего списка укажите, как в деревьях истинности разбирать уравнения видов $A~op~B = 1$ и $A~op~B = 0$: стрелка Пирса $\downarrow$, штрих Шеффера $\mid$, равносильность $\leftrightarrow$, сумма по модулю 2 $\oplus$ (она же исключающее или). Если какие-то из них незнакомы и не получится найти определение, напишите.
    \item $(p \leftrightarrow q) \leftrightarrow r \equiv p \leftrightarrow (q \leftrightarrow r)$ (используйте правила из задания \ref*{rules}).
\end{enumerate}

\section{Натуральная дедукция для логики высказываний}
Доказать с помощью натуральной дедукции:
\begin{enumerate}
    \item $(p \land q) \land r \equiv p \land (q \land r)$.
    \item $p \to q \equiv \neg q \to \neg p$.
    \item $p \to q \land r \equiv (p \to q) \land (p \to r)$.
    \item $p \equiv \neg \neg p$.
    \item $p \lor q \to r \equiv (p \to r) \land (q \to r)$.
    \item Законы де Моргана (2 в обоих направлениях).
    \item $\vdash p \lor \neg p$ (использовать RAA и закон де Моргана для отрицания дизъюнкции).
    \item \label{to_equiv} $p \to q \equiv \neg p \lor q$.
    \item $p \land (q \lor r) \vdash (p \land q) \lor r$.
    \itemhard \label{h21} $\vdash ((p \to q) \to p) \to p$ (есть \hyperref[hints]{подсказка}).
    \item \label{rules2} Составить правила введения и исключения для равносильности, суммы по модулю 2, стрелки Пирса и штриха Шеффера. Подсказка: выразите через операции, для которых эти правила уже знаете, и скомбинируйте их.
    \item[] Используя эти правила:
    \item $p \downarrow q \vdash p \mid q$.
    \itemhard $(p \leftrightarrow q) \leftrightarrow r \equiv p \leftrightarrow (q \leftrightarrow r)$.
\end{enumerate}

\section{Формализация утверждений в логике предикатов}
Перевести на язык логики предикатов:
\begin{enumerate}
    \item Мне скучно.
    \item Иванов и Петров играют в шахматы.
    \item Иванов и Петров слушают лекцию.
    \item Каждый, кто упорно работает, добивается успеха.
    \item Слон Бимбо больше собаки Ланды.
    \item Кошки бывают только белые и серые.
    \item Среднее арифметическое любых двух чисел больше иx среднего геометрического (не используйте операции деления и извлечения корня, так как они не везде определены).
    \item Уравнение $x^2 - 3x + 2 = 0$ не имеет решений.
    \item Функция $f$ непрерывна в точке $a$ (используйте определение предела через $\varepsilon$ и $\delta$).
    \item Точки $A, B, C$ являются вершинами равнобедренного треугольника.
    \item Функция $f$ принимает в том числе такие комплексные значения, которые не являются действительными.
    \item У каждого положительного действительного числа есть ровно один положительный квадратный корень.
    \item Число $x$ простое (для групп, где не сделали на занятии).
    \item Есть сколько угодно большие простые числа.
    \item Последовательность $a_0, a_1, \ldots$ имеет более одной предельной точки.
\end{enumerate}

\section{Деревья истинности для логики предикатов}
Доказать или опровергнуть с помощью деревьев истинности:
\begin{enumerate}
    \item $\exists x~(P(x) \to \forall y~P(y))$
    \item $\forall x~(P(x) \lor Q(x)) \vdash (\exists x~P(X)) \lor (\forall x~Q(x))$
    \item $\exists x~P(x), \forall x~(P(x) \to Q(x)) \vdash \exists x~Q(x)$
    \item $\neg \exists x~(P(x) \to Q(x)) \vdash (\exists x~A(x)) \land (\forall x~Q(x))$
    \item $\forall x \exists y~P(x,y), \exists x \forall y~Q(x,y) \vdash \exists x \exists y~(P(x,y) \land Q(x,y))$
    \item[] Следующие формализовать как секвенции с двумя посылками:
    \item Не все политики мошенники. Все мошенники умны. Значит, некоторые политики глупы.
    \item Те, кто что-то учил, решили некоторые задачи. Андрей не решил ни одной. Значит, он не учил ничего.
    \item Если бинарное отношение транзитивно и симметрично, то оно рефлексивно (здесь квантора по бинарным отношениям нет, просто обозначьте его как $R(x,y)$).
    \item[] Задачи с равенством и с функциями:
    \item $\forall x \exists! y~f(x)=y$
    \item Эквивалентность двух формализаций $\exists! x~P(x)$:\\$\exists x~(P(x) \land \forall y~(P(y) \to x=y)) \equiv (\exists x~P(x)) \land \forall y \forall z~(P(y) \land P(z) \to y=z)$
    \itemhard $\forall x \forall y~(P(f(x), y) \lor Q(x,y)),~\forall x \forall y~(\neg P(x,g(y)) \lor Q(x,y)) \vdash \exists x \exists y~Q(x,y)$
\end{enumerate}

\section{Натуральная дедукция для логики предикатов}
Доказать с помощью натуральной дедукции:
\begin{enumerate}
    \item $(\forall x~P(x)) \to (\exists x~P(x))$
    \item $\forall x~P(x), \exists x~(P(x) \to Q(x)) \vdash \exists x~Q(x)$
    \item $\neg \forall x~P(x) \equiv \exists x~\neg P(x)$.
    \item $\exists x~(P(x) \land Q(x)) \vdash (\exists x~P(x)) \land (\exists x~Q(x))$
    \item $\forall x~(P(x) \lor Q(x)) \vdash (\exists x~P(x)) \lor (\forall x~Q(x))$
    \item $\forall x \exists y~P(x,y),~\exists x \forall y~Q(x,y) \vdash \exists x \exists y~(P(x,y) \land Q(x,y))$
    \item Убедитесь, что $\exists x~P(x) \vdash \forall x~P(x)$ нельзя доказать и постройте контрмодель.
    \item $\exists x~P(f(x)) \vdash \exists x~P(x)$
    \item $\forall x \exists! y~f(x)=y$ (если не получится, сделайте упрощённый вариант с $\exists y$)
    \item $\forall x \forall y~P(f(x), y),~\forall x \forall y~\neg P(x,g(y)) \vdash \bot$
    \itemhard Эквивалентность двух формализаций $\exists! x~P(x)$:\\$\exists x~(P(x) \land \forall y~(P(y) \to x=y)) \equiv (\exists x~P(x)) \land \forall y \forall z~(P(y) \land P(z) \to y=z)$
    \itemhard \label{h51} $\exists x~(P(x) \to \forall y~P(y))$
    \itemhard \label{h52} $\forall x~(R(x,y) \to R(y,x)),~~\forall x \forall y \forall z~(R(x,y) \land R(y,z) \to R(x,z)),~~\forall x \exists y~R(x,y) \vdash \forall x~R(x,x)$
\end{enumerate}
Для~\ref*{h51} и~\ref*{h52} есть \hyperref[hints]{подсказки}.

\section{Теория множеств: функции}
\begin{enumerate}
    \item Проверьте, является ли $f: \N \to \Z,~f(n)=n^2+n+1$ а) вложением, б) наложением, в) биекцией.
    \item Проверьте, является ли $f: \Q_{\geq 0} \to \Q_{\geq 0},~f(x)=x^2$ а) вложением, б) наложением, в) биекцией.
    \item Проверьте, является ли $f: \R \to \R,~f(x)=x + \sin x$ а) вложением, б) наложением, в) биекцией.
    \item Пусть $A, B$ множества. Рассмотрим $i: A \to B,~i(x)=x$. Для каких $A,B$ $i$ будет а) функцией? б) вложением? в) наложением? г) биекцией?
    \item Пусть $A, B$ множества. Рассмотрим $f: A \times B \to B \times A,~f(x,y)=(y,x)$. Показать, что это всегда биекция.
    \item Пусть $A$ множество. Рассмотрим $f: A \to \mathcal{P}(A),~f(x)=\{x\}$. Для каких $A$ $f$ будет а) функцией? б) вложением? в) наложением? г) биекцией?
    \item Пусть $A,B,C$ множества, $f: A \to B, g: B \to C$ функции. Докажите:
        \begin{enumerate}
            \item Если $f$ и $g$ вложения, то $g \circ f = x \mapsto g(f(x))$ вложение.
            \item Если $f$ и $g$ наложения, то $g \circ f$ наложение.
        \end{enumerate}
    \item Пусть $A,B,C,D$ множества, $f: A \to B, g: C \to D$ вложения.
        \begin{enumerate}
            \item Если $A \cup B = C \cup D = \varnothing$, найдите вложение $h: A \cup B \to C \cup D$. Что изменится без этого дополнительного условия?
            \item Найдите вложение $h: A \times B \to C \times D$.
            \item Покажите, что если $f$ и $g$ наложения (и не обязательно вложения), то $h$ из обоих предыдущих пунктов будут наложениями.
        \end{enumerate}
    \item \label{h61} Пусть $A$ множество. Найти биекцию между $\mathcal{P}(A)$ и $A \to \{0,1\}$.
    \itemhard Пусть $A,B$ непустые множества, $f: A \to B$ функция. Докажите:
        \begin{enumerate}
            \item $\exists g: B \to A~\forall x: A~g(f(x))=x$ ($g$ левая обратная к $f$) $\iff$ $f$ вложение.
            \item $\exists g: B \to A~\forall y: B~f(g(y))=y$ ($g$ правая обратная к $f$) $\iff$ $f$ наложение.
            \item Если $g,h: B \to A$, $g$ левая обратная к $f$, и $h$ правая обратная к $f$, то $g=h$.
        \end{enumerate}
    \itemhard \label{h62} Пусть $A,B,C$ множества. Найдите биекцию между $A \to (B \to C)$ и $A \times B \to C$.
    \itemhard Определение упорядоченной пары по Куратовскому: $(a,b) = \{\{a\},\{a,b\}\}$.
        \begin{enumerate}
            \item Докажите, что $(a,b)=(c,d) \iff a=c \land b=d$. Нужно учесть, что любые из $a,b,c,d$ могут быть равны между собой.
            \item Определим $pr_1(A) = \bigcup \bigcap A$. Докажите, что $pr_1((a,b))=a$. (Определение второй проекции существенно сложнее.)
        \end{enumerate}
\end{enumerate}
Для~\ref*{h61} и~\ref*{h62} есть \hyperref[hints]{подсказки}.

\section{Теория множеств: мощности}
\begin{enumerate}
    \item Найдите мощность множества всех многочленов с рациональными коэффициентами.
    \item Найдите мощность множества всех алгебраических чисел (действительных корней многочленов с рациональными коэффициентами). Используйте предыдущую задачу.
    \item Докажите, что мощность любого отрезка равна мощности любого интервала.
    \item Докажите, что мощности из предыдущей задачи равны $|\R|$.
    \item \label{h71} Найдите мощность множества всех прямых на плоскости.
    \item Найдите мощность множества всех невырожденных треугольников на плоскости.
    \item Найдите мощность множества иррациональных чисел.
    \item Найдите мощность множества строго возрастающих бесконечных последовательностей натуральных чисел.
    \item Найдите мощность множества строго убывающих бесконечных последовательностей натуральных чисел.
    \item Найдите мощность множества нестрого убывающих бесконечных последовательностей натуральных чисел.
    \item Докажите, что $|\R| = |\N \to \{0,1\}|$.
    \item Найдите мощность множества всех функций $\R \to \R$.
    \item Найдите мощность множества всех вложений $\N \to \N$.
    \itemhard Докажите, что $\continuum^2 = \continuum$ (построением явной биекции).
    \itemhard \label{h72} Докажите, что мощность множества непрерывных функций $\R \to \R$ равна $\continuum$.
\end{enumerate}
Для~\ref*{h71} и~\ref*{h72} есть \hyperref[hints]{подсказки}.

\section{Теория множеств: ординалы}
\begin{enumerate}
    \item Докажите:
        \begin{enumerate}
            \item Если $A$ и $B$ частично упорядочены, то $A + B$ и $A \times B$ частично упорядочены.
            \item Если $A$ и $B$ линейно упорядочены, то $A + B$ и $A \times B$ линейно упорядочены.
            \item Если $A$ и $B$ фундированы, то $A + B$ и $A \times B$ фундированы.
        \end{enumerate}
    \item Приведите к нормальной форме Кантора: $(\omega+2)(\omega+1)\omega$.
    \item Приведите к нормальной форме Кантора: $(\omega+2)^3$.
    \item Приведите к нормальной форме Кантора: $(\omega+1)^{\omega+1}$.
    \item Найдите контрпример к утверждению $(\alpha + \beta) \cdot \gamma = \alpha \cdot \gamma + \beta \cdot \gamma$.
    \item Докажите, что $(\alpha+\beta)+\gamma = \alpha+(\beta+\gamma)$.
    \item Докажите, что если $\alpha < \beta$, то 
        \begin{enumerate}
            \item $\gamma + \alpha < \gamma + \beta$.
            \item $\alpha + \gamma \leq \beta + \gamma$ (но не обязательно $<$).
        \end{enumerate}
    \itemhard Докажите, что $\alpha + \beta = \beta \Leftrightarrow \alpha \cdot \omega \leq \beta$.
    \itemhard Докажите, что:
    \begin{enumerate}
        \item Ординалы вида $\omega^\alpha$ нельзя представить как сумму двух меньших ординалов.
        \item Любые другие положительные ординалы можно так представить.
    \end{enumerate}
\end{enumerate}

\pagebreak
\section*{Подсказки}
\label{hints}

2.\ref{h21}: можно использовать задачу \ref{to_equiv} или закон исключённого третьего.

5.\ref{h51}: можно использовать закон исключённого третьего для $\forall y~P(y)$.

5.\ref{h52}: так как $\forall x \exists y~R(x,y)$, то для этого $y$ также верно $R(y,x)$, а из $R(x,y)$ и $R(y,x)$ заключаем $R(x,x)$.

6.\ref{h61}: $f: \mathcal{P}(A) \to (A \to \{0,1\}),~f(B) = x \mapsto \begin{cases} 1, & x \in B \\ 0, & x \notin B \end{cases}$. Остаётся доказать, что это действительно биекция (можно найти обратную функцию).

6.\ref{h62}: $f: (A \to (B \to C)) \to (A \times B \to C),~f(g) = (x,y) \mapsto g(x)(y)$. Остаётся доказать, что это действительно биекция (можно найти обратную функцию).

7.\ref{h71}: точно ли вы задали произвольные прямые, включая параллельные обеим осям?

7.\ref{h72}: если две непрерывные функции $\R \to \R$ совпадают на рациональных точках, то они совпадают на всей прямой (почему?).

\end{document}
