\documentclass[10pt]{beamer}
\usepackage[utf8]{inputenc}
\usepackage[T1,T2A]{fontenc}
\usepackage[russian]{babel}
\usepackage{color}
\usepackage{calc}
\usepackage{graphicx}
\usepackage{epstopdf}
\usepackage{hyperref}
\hypersetup{unicode,colorlinks}
\usepackage{csquotes}
\usepackage{upquote}
\usepackage{cprotect}
\usetheme[progressbar=head,numbering=fraction,block=fill]{metropolis}
\usepackage{dejavu}
\usepackage{etoolbox}
\usepackage{bm}
\usepackage[ND]{prftree}
\usepackage[tableaux]{prooftrees}
\usepackage{mathtools} % for bigtimes
\usepackage{diagbox}
\usepackage{verbatimbox}
\usepackage{fitch}
\usepackage{enumitem}
%\usepackage[defaultsans]{droidsans}
%\usepackage[defaultmono]{droidmono}
%\makeatletter
%\input droid/t1fdm.fd
%\makeatother

\makeatletter
\DeclareRobustCommand*{\bfseries}{%
    \not@math@alphabet\bfseries\mathbf
    \fontseries\bfdefault\selectfont
    \boldmath
}
\makeatother

\makeatletter
\chardef\straightquote@code=\catcode`'
\chardef\backquote@code=\catcode``
\catcode`'=\active \catcode``=\active
\patchcmd{\@noligs}
{\textasciigrave}
{\fixedtextasciigrave}
{}{}
\newcommand{\fixedtextasciigrave}{%
    \makebox[.5em]{\fontencoding{TS1}\fontfamily{fvs}\selectfont\textasciigrave}% Vera Sans
}
\catcode\lq\'=\straightquote@code
\catcode\lq\`=\backquote@code
\makeatletter

\setbeamercovered{again covered={\opaqueness<1->{25}}}

\forestset{close with=\ensuremath{\bigtimes}}

\newcommand{\customframefont}[1]{
    \setbeamertemplate{itemize/enumerate body begin}{#1}
    \setbeamertemplate{itemize/enumerate subbody begin}{#1}
}

\NewEnviron{framefont}[1]{
    \customframefont{#1} % for itemize/enumerate
    {#1 % For the text outside itemize/enumerate
        \BODY
    }
    \customframefont{\normalsize}
}


\makeatletter
\DeclareRobustCommand{\rvdots}{%
    \vphantom{\int\limits^x}\smash[t]{\vdots}
}
\makeatother

\newcommand{\N}{\mathbb{N}}
\newcommand{\Z}{\mathbb{Z}}
\newcommand{\Q}{\mathbb{Q}}
\newcommand{\R}{\mathbb{R}}
\newcommand{\pow}[1]{\mathcal{P}({#1})}
\newcommand{\continuum}{\mathfrak{c}}

\author{Алексей Романов}
\subtitle{Математическая логика и теория алгоритмов}
%\logo{}
\institute{МИЭТ}
\subject{Математическая логика и теория алгоритмов}
%\setbeamercovered{transparent}
%\setbeamertemplate{navigation symbols}{}


\title{Логика предикатов\\Деревья истинности}
\date{\today}

\begin{document}
\begin{frame}[plain]
\maketitle
\end{frame}

\begin{frame}
\frametitle{Доказательства в логике предикатов}
\begin{itemize}
    \item В логике предикатов естественно встаёт вопрос:
    \item Дана формула, тождественно истинна ли она?
    \item Что значит \enquote{Формула тождественно истинна}? \pause 
    \begin{itemize}
        \item Замкнутая: истинна на всех моделях её сигнатуры. \pause
        \item Со свободными переменными: можем навесить квантор всеобщности. \pause
    \end{itemize}
    \item Варианты: эквивалентны ли две формулы? Является ли формула теоремой данной теории (это сложнее!)?
    \item[]
    \item Как это можно проверить?
    \item Нельзя просто перечислить все модели: \pause их бесконечно (и даже несчётно) много!
    \item Можно расширить деревья истинности и натуральную дедукцию. 
    \item Сегодня деревья истинности.
\end{itemize}
\end{frame}

\begin{frame}
\frametitle{Правила деревьев истинности}
\begin{itemize}
    \item Правила для $\land$/$\lor$/$\to$/$\neg$ (типы $\alpha$ и $\beta$) сохраняются.
    \item Добавляются правила для формул с кванторами.
    \item Есть несколько способов, рассмотрим простейший.
    \item Добавляется бесконечное множество \emph{параметров} $a_1,a_2,\ldots$, не зависящих от сигнатуры.
    \item Два новых типа формул: \pause
    \addvbuffer[0.5em 0.5em]{
    \begin{tabular}{|c | c || c | c |}
    \hline
    $\gamma$ & $\gamma(t)$ & $\delta$ & $\delta(t)$ \\
    \hline
    $\forall v ~ A(v) = 1$ & $A(t) = 1$ & $\forall v ~ A(v) = 0$ & $A(t) = 0$ \\
    $\exists v ~ A(v) = 0$ & $A(t) = 0$ & $\exists v ~ A(v) = 1$ & $A(t) = 1$ \\
    \hline
    \end{tabular}}
    \item Правила для них: \pause
\begin{columns}
    \begin{column}{0.5\textwidth}
        \begin{center}
            \begin{tableau}{not line numbering, single branches}
                [{\gamma} [\gamma(t)]]
            \end{tableau}\\
            \vspace{0.5em}
            $t$ --- замкнутый терм,\\ можно раскрыть много раз
        \end{center}
    \end{column} \pause
    \begin{column}{0.5\textwidth}
        \begin{center}
            \begin{tableau}{not line numbering, single branches}
                [{\delta} [\delta(a)]]
            \end{tableau}\\
        \vspace{0.5em}
        $a$ --- новый (для ветви) параметр
        \end{center}
    \end{column}
\end{columns}
\end{itemize}
\end{frame}

\begin{frame}
\frametitle{Пояснения к правилам}
\begin{itemize}
    \item Смысл правила $\gamma$: утверждение, верное для всех объектов, верно в том числе для значения $t$.
    \item Смысл правила $\delta$: мы даём название $a$ тому объекту, существование которого утверждается. \pause 
    \item Обратите внимание, что $\forall x ~ A(x) = 0$ читается как \pause $(\forall x ~ A(x)) = 0$, а не $\forall x ~ (A(x) = 0)$. \pause
    \item \emph{Замкнутый терм} не содержит переменных. В случае без константных и функциональных символов это просто параметр. \pause
    \item $\gamma$-формулы не помечаются $\checkmark$ при раскрытии, чтобы показать, что они могут быть повторно раскрыты. Только использованным термом.
    \item Для $\delta$-формул параметр можно указать рядом с $\checkmark$, чтобы легко увидеть использованные параметры.
\end{itemize}
\end{frame}

\begin{frame}
\frametitle{Пример дерева истинности}
\begin{itemize}
    \item Первый пример: одно из правил де Моргана для кванторов, $\neg (\forall x ~ P(x)) \to (\exists x ~ \neg P(x))$.
    \item[] 
\only<+>{
    \begin{tableau}{
    }
    [{\neg (\forall x ~ P(x)) \to (\exists x ~ \neg P(x)) = 0}, just=Дано]
    \end{tableau}
}
\only<+>{
    \begin{tableau}{
    }
    [{\neg (\forall x ~ P(x)) \to (\exists x ~ \neg P(x)) = 0}, just=Дано, checked
    [{\neg (\forall x ~ P(x)) = 1}, just=1, checked
    [{\exists x ~ \neg P(x) = 0}, just = 1
    [{\forall x ~ P(x) = 0}, just=2
    ]]]]
    \end{tableau}
}
\only<+>{
    \begin{tableau}{
        }
        [{\neg (\forall x ~ P(x)) \to (\exists x ~ \neg P(x)) = 0}, just=Дано, checked
        [{\neg (\forall x ~ P(x)) = 1}, just=1, checked
        [{\exists x ~ \neg P(x) = 0}, just = 1
        [{\forall x ~ P(x) = 0}, just=2, checked=a_1
        [{P(a_1) = 0}, just=4, checked
        ]]]]]
    \end{tableau}
}
\onslide<+->
    \begin{tableau}{
        }
        [{\neg (\forall x ~ P(x)) \to (\exists x ~ \neg P(x)) = 0}, just=Дано, checked
        [{\neg (\forall x ~ P(x)) = 1}, just=1, checked
        [{\exists x ~ \neg P(x) = 0}, just = 1, subs=a_1
        [{\forall x ~ P(x) = 0}, just=2, checked=a_1
        [{P(a_1) = 0}, just=4, checked
        [{\neg P(a_1) = 0}, just=3, checked
        [{P(a_1) = 1}, just=6, checked, close={5,7}
        ]]]]]]]
    \end{tableau}
\item Дерево закрылось, значит формула $\neg (\forall x ~ P(x)) \to (\exists x ~ \neg P(x))$ \onslide<+-> тождественно истинна.
\end{itemize}
\end{frame}

\begin{frame}
\frametitle{Порядок действий}
\begin{itemize}
    \item Заметьте, что если сначала раскрыли бы строку 3 с параметром $a_1$, то в строке 4 пришлось бы использовать новый параметр $a_2$.
    \item И противоречия не получится, пока не раскроем строку 3 с $a_2$. \pause
    \item Также правила позволяют раскрыть строку 3 сколько угодно раз с разными параметрами и не дойти до строки 4.
    \item Тогда опять не получим противоречия. \pause
    \item Поэтому порядок раскрытия строк теперь важен для результата. \pause
    \item Сначала $\alpha$ и $\delta$, потом $\beta$, потом $\gamma$. \pause
    \item Это не единственно возможный, но достаточно простой.
\end{itemize}
\end{frame}


\begin{frame}
\frametitle{Пример дерева истинности (2)}
\begin{itemize}
    \item Ещё пример: $\forall x ~ P(x) \land Q(x) \vdash (\forall x ~ P(x)) \land (\forall x ~ Q(x))$. \\ Как думаете, тождественно истинна или нет?
    \item[] 
    \only<+>{
        \begin{tableau}{
            }
            [{\forall x ~ P(x) \land Q(x) = 1}, just=Дано
            [{(\forall x ~ P(x)) \land (\forall x ~ Q(x)) = 0}, just=Дано]]
        \end{tableau}
    }
    \only<+>{
    \begin{tableau}{
        }
        [{\forall x ~ P(x) \land Q(x) = 1}, just=Дано
        [{(\forall x ~ P(x)) \land (\forall x ~ Q(x)) = 0}, just=Дано, checked
        [{\forall x ~ P(x) = 0}, just = 2
        ]
        [{\forall x ~ Q(x) = 0}, just = 2
        ]]]
        \end{tableau}
    }
    \only<+>{
        \begin{tableau}{
            }
        [{\forall x ~ P(x) \land Q(x) = 1}, just=Дано
        [{(\forall x ~ P(x)) \land (\forall x ~ Q(x)) = 0}, just=Дано, checked
        [{\forall x ~ P(x) = 0}, just = 2, checked=a_1
        [{P(a_1) = 0}, just=3, checked
        ]]
        [{\forall x ~ Q(x) = 0}, just = 2, checked=a_1
        [{Q(a_1) = 0}, just=3, checked
        ]
        ]]]
        \end{tableau}
    }
    \only<+>{
        \begin{tableau}{
            }
        [{\forall x ~ P(x) \land Q(x) = 1}, just=Дано, subs=a_1
        [{(\forall x ~ P(x)) \land (\forall x ~ Q(x)) = 0}, just=Дано, checked
        [{\forall x ~ P(x) = 0}, just = 2, checked=a_1
        [{P(a_1) = 0}, just=3, checked
        [{P(a_1) \land Q(a_1) = 1}, just=1
        ]]]
        [{\forall x ~ Q(x) = 0}, just = 2, checked=a_1
        [{Q(a_1) = 0}, just=3, checked
        [{P(a_1) \land Q(a_1) = 1}, just=1
        ]]
        ]]]
        \end{tableau}
    }
    \onslide<+->
    \begin{tableau}{
        }
        [{\forall x ~ P(x) \land Q(x) = 1}, just=Дано, subs=a_1
        [{(\forall x ~ P(x)) \land (\forall x ~ Q(x)) = 0}, just=Дано, checked
        [{\forall x ~ P(x) = 0}, just=2, checked=a_1
        [{P(a_1) = 0}, just=3, checked
        [{P(a_1) \land Q(a_1) = 1}, just=1, checked
        [{P(a_1) = 1}, just=5, checked, close={4,6}
        ]]]]
        [{\forall x ~ Q(x) = 0}, just=2, checked=a_1
        [{Q(a_1) = 0}, just=3, checked
        [{P(a_1) \land Q(a_1) = 1}, just=1, checked
        [{Q(a_1) = 1}, just=5, checked, close={4,6}
        ]]]
        ]]]
    \end{tableau}
    \item Дерево закрылось, значит секвенция \onslide<+-> тождественно истинна. \onslide<3->
    \item В строке 3 можем использовать $a_1$ в обеих ветвях. \onslide<4->
    \item Когда раскрываем строку 1, под ней две открытые ветви, результат пишем в обе.
\end{itemize}
\end{frame}

\begin{frame}
\frametitle{Пример дерева истинности (3)}
\begin{itemize}
    \item $\forall x ~ P(x) \lor Q(x) \vdash (\forall x ~ P(x)) \lor (\forall x ~ Q(x))$. \\ Как думаете, тождественно истинна или нет?
    
    \only<+>{
        \begin{tableau}{
            }
            [{\forall x ~ P(x) \lor Q(x) = 1}, just=Дано
            [{(\forall x ~ P(x)) \lor (\forall x ~ Q(x)) = 0}, just=Дано
            ]]
        \end{tableau}
    }
    
    \only<+>{
        \begin{tableau}{
            }
            [{\forall x ~ P(x) \lor Q(x) = 1}, just=Дано
            [{(\forall x ~ P(x)) \lor (\forall x ~ Q(x)) = 0}, just=Дано, checked
            [{\forall x ~ P(x) = 0}, just = 2
            [{\forall x ~ Q(x) = 0}, just = 2
            ]]]]
        \end{tableau}
    }
    
    \only<+>{
        \begin{tableau}{
            }
            [{\forall x ~ P(x) \lor Q(x) = 1}, just=Дано
            [{(\forall x ~ P(x)) \lor (\forall x ~ Q(x)) = 0}, just=Дано, checked
            [{\forall x ~ P(x) = 0}, just = 2, checked=a_1
            [{\forall x ~ Q(x) = 0}, just = 2, checked=a_2
            [{P(a_1) = 0}, just=3, checked
            [{Q(a_2) = 0}, just=4, checked
            ]]]]]]
        \end{tableau}
    }
    
        \only<+>{
        \begin{tableau}{
            }
            [{\forall x ~ P(x) \lor Q(x) = 1}, just=Дано, subs={a_1}
            [{(\forall x ~ P(x)) \lor (\forall x ~ Q(x)) = 0}, just=Дано, checked
            [{\forall x ~ P(x) = 0}, just = 2, checked=a_1
            [{\forall x ~ Q(x) = 0}, just = 2, checked=a_2
            [{P(a_1) = 0}, just=3, checked
            [{Q(a_2) = 0}, just=4, checked
            [{P(a_1) \lor Q(a_1) = 1}, just=1
            ]]]]]]]
        \end{tableau}
    }

    \only<+>{
    \begin{tableau}{
        }
        [{\forall x ~ P(x) \lor Q(x) = 1}, just=Дано, subs={a_1}
        [{(\forall x ~ P(x)) \lor (\forall x ~ Q(x)) = 0}, just=Дано, checked
        [{\forall x ~ P(x) = 0}, just = 2, checked=a_1
        [{\forall x ~ Q(x) = 0}, just = 2, checked=a_2
        [{P(a_1) = 0}, just=3, checked
        [{Q(a_2) = 0}, just=4, checked
        [{P(a_1) \lor Q(a_1) = 1}, just=1, checked
        [{P(a_1) = 1}, just={7 (ветвь закончена?)}, checked, close={5,8}
        ]
        [{Q(a_1) = 1}, just={7 (ветвь закончена?)}, checked
        ]]]]]]]]
    \end{tableau}
}

    \only<+>{
        \begin{tableau}{
            }
        [{\forall x ~ P(x) \lor Q(x) = 1}, just=Дано, subs={a_1,a_2}
[{(\forall x ~ P(x)) \lor (\forall x ~ Q(x)) = 0}, just=Дано, checked
[{\forall x ~ P(x) = 0}, just = 2, checked=a_1
[{\forall x ~ Q(x) = 0}, just = 2, checked=a_2
[{P(a_1) = 0}, just=3, checked
[{Q(a_2) = 0}, just=4, checked
[{P(a_1) \lor Q(a_1) = 1}, just=1, checked
[{P(a_1) = 1}, just=7, checked, close={5,8}
]
[{Q(a_1) = 1}, just=7, checked
[{P(a_2) \lor Q(a_2) = 1}, just=1
]]]]]]]]]
        \end{tableau}
    }

    \onslide<+->
    \begin{tableau}{
        }
        [{\forall x ~ P(x) \lor Q(x) = 1}, just=Дано, subs={a_1,a_2}
        [{(\forall x ~ P(x)) \lor (\forall x ~ Q(x)) = 0}, just=Дано, checked
        [{\forall x ~ P(x) = 0}, just = 2, checked=a_1
        [{\forall x ~ Q(x) = 0}, just = 2, checked=a_2
        [{P(a_1) = 0}, just=3, checked
        [{Q(a_2) = 0}, just=4, checked
        [{P(a_1) \lor Q(a_1) = 1}, just=1, checked
        [{P(a_1) = 1}, just=7, checked, close={5,8}
        ]
        [{Q(a_1) = 1}, just=7, checked
        [{P(a_2) \lor Q(a_2) = 1}, just=1, checked
        [{P(a_2) = 1}, just=9, checked
        ]
        [{Q(a_2) = 1}, just=9, checked, close={6,10}
        ]
        ]]]]]]]]]
    \end{tableau}
    \item Дерево не закрылось!
\end{itemize}
\end{frame}

\begin{frame}
\frametitle{Пример дерева истинности (3)}
\begin{itemize}
    \item Получили открытую ветвь, в которой: 
    \begin{itemize}
        \item все $\alpha$-, $\beta$-, $\delta$- и $\neg$-формулы раскрыты;
        \item все $\gamma$-формулы раскрыты со всеми параметрами (и хотя бы с одним).
    \end{itemize}
    \item Такая ветвь называется законченной. \pause
    \item В нашем случае атомы в ней $P(a_1) = 0$, $Q(a_2) = 0$, $Q(a_1) = 1$, $P(a_2) = 1$.
    \item По ним строим модель. В ней два объекта, которые так и обозначим: $a_1$ и $a_2$. \\ \pause
    \begin{tabular}{|c | c c |}
    \hline
    $x$ & $a_1$ & $a_2$ \\
    \hline
    $P(x)$ & $0$ & $1$ \\
    \hline
    $Q(x)$ & $1$ & $0$ \\
    \hline
    \end{tabular} \pause
    \item Что делать, если бы в ветви не было атома c $P(a_1)$ (например)? \pause
    \item Можем поставить туда любое значение. \pause
    \item Снова видим, что $x$, связанные разными кванторами, по сути разные переменные (строки 5-6).
\end{itemize}
\end{frame}

\begin{frame}
\frametitle{Пример дерева истинности (4)}
\begin{itemize}
    \item $\forall x \exists y ~ P(x, y) \vdash \exists y \forall x ~ P(x, y)$. \\ Как думаете, тождественно истинна или нет?
    \item[] 
\only<+>{
    \begin{tableau}{
        }
        [{\forall x \exists y ~ P(x, y) = 1}, just=Дано
        [{\exists y \forall x ~ P(x, y) = 0}, just=Дано
        ]]
    \end{tableau}
}

    \only<+>{
    \begin{tableau}{
        }
        [{\forall x \exists y ~ P(x, y) = 1}, just=Дано, subs={a_1}
        [{\exists y \forall x ~ P(x, y) = 0}, just=Дано
        [{\exists y ~ P(a_1, y) = 1}, just=1, checked=a_2
        [{P(a_1, a_2) = 1}, just=3, checked
        ]]]]
    \end{tableau}
}

    \only<+>{
\begin{tableau}{
    }
    [{\forall x \exists y ~ P(x, y) = 1}, just=Дано, subs={a_1}
    [{\exists y \forall x ~ P(x, y) = 0}, just=Дано, subs={a_2}
    [{\exists y ~ P(a_1, y) = 1}, just=1, checked=a_2
    [{P(a_1, a_2) = 1}, just=3, checked
    [{\forall x ~ P(x, a_2) = 0}, just=2, checked=a_3
    [{P(a_3, a_2) = 0}, just={5 (ветвь закончена?)}, checked
    ]]]]]]
\end{tableau}
}

    \onslide<+->
    \begin{tableau}{
        }
        [{\forall x \exists y ~ P(x, y) = 1}, just=Дано, subs={a_1,a_3,\ldots}
        [{\exists y \forall x ~ P(x, y) = 0}, just=Дано, subs={a_2,\ldots}
        [{\exists y ~ P(a_1, y) = 1}, just=1, checked=a_2
        [{P(a_1, a_2) = 1}, just=3, checked
        [{\forall x ~ P(x, a_2) = 0}, just=2, checked=a_3
        [{P(a_3, a_2) = 0}, just=5, checked
        [{\exists y ~ P(a_3, y) = 1}, just=1, checked=a_4
        [{\ldots}
        ]]]]]]]]
    \end{tableau}
    \onslide<+->
    \item На этот раз ситуация сложнее предыдущей.
    \item Противоречия нет, но и законченной ветви (в смысле прошлого слайда) тоже.
    \item По крайней мере после любого конечного числа шагов.
    \onslide<+->
    \item Наверное, легко \emph{увидеть}, что противоречия так и не получим, но как это \emph{доказать}?
\end{itemize}
\end{frame}

\begin{frame}
\frametitle{Бесконечные контрмодели}
\begin{itemize}
    \item В нашей ветви видны некоторые значения предиката:
    \item[] \begin{tabular}{c | c c c c c}
        \diagbox[height=1.5\line]{x}{y} & $a_1$ & $a_2$ & $a_3$ & $a_4$ & \ldots \\ \hline
        $a_1$ & ? & 1 & ? & ? & \\
        $a_2$ & ? & ? & ? & ? & \\
        $a_3$ & ? & 0 & ? & 1 & \\
        $a_4$ & ? & ? & ? & ? & \\
        \ldots & & & & & \\ 
    \end{tabular}
    \item Как в терминах этой таблицы выглядят:
    \begin{itemize}
        \item $\forall x \exists y ~ P(x, y) = 1$? \pause В каждой строке есть 1. \pause
        \item $\exists y \forall x ~ P(x, y) = 0$? \pause Нет столбца, где все 1. 
        \item[] Помните, что это $(\exists y \forall x ~ P(x, y)) = 0$, а не $\exists y \forall x ~ (P(x, y) = 0)$!
    \end{itemize}
\end{itemize}
\end{frame}

\begin{frame}
\frametitle{Бесконечные контрмодели}
\begin{itemize}
    \item Видно, что расставить 0 и 1 на место ? произвольно нельзя, но вот вариант:
    \item[] {\small \begin{tabular}{c | c c c c c}
        \diagbox[height=1.5\line]{x}{y} & $a_1$ & $a_2$ & $a_3$ & $a_4$ & \ldots \\ \hline
        $a_1$ & 0 & 1 & 0 & 0 & \\
        $a_2$ & 0 & 0 & 1 & 0 & \\
        $a_3$ & 0 & 0 & 0 & 1 & \\
        $a_4$ & 0 & 0 & 0 & 0 & \\
        \ldots & & & & & \\ 
    \end{tabular}}
    \pause
    \item[]
    \item Мы нашли бесконечный контрпример для этой секвенции, а есть ли конечный? Напомню условия:
    \item В каждой строке есть 1; нет столбца, где все 1. 
    \pause
    \item Конечно, есть. Например:
    \item[] {\small \begin{tabular}{c | c c |}
        \diagbox[height=1.5\line]{x}{y} & $a_1$ & $a_2$ \\ \hline
        $a_1$ & 1 & 0 \\
        $a_2$ & 0 & 1 \\
        \hline
    \end{tabular}}
\end{itemize}
\end{frame}

\begin{frame}
    \frametitle{Теорема о корректности}
    \begin{itemize}
        \item Теорема: если для $A=0$ есть закрытое дерево истинности, то $A$ тождественно истинна.
        \item Общий ход доказательства тот же, что для логики высказываний. Напомню:
        \pause
        \item Доказательство от противного. Если $A$ не тождественно истинна, то $A=0$ выполнимо.
        \item Значит, в исходном дереве единственная ветвь выполнима.
        \item Лемма: если в дереве есть выполнимая ветвь, после применения правил такая ветвь тоже есть.
        \item Значит, сколько не применяем правила к исходному дереву, на каждом шаге есть выполнимая ветвь.
        \item Закрытая ветвь не может быть выполнима, значит, на каждом шаге есть открытая ветвь.
        \item Значит, дерево не может быть закрыто.
        \pause
        \item Остаётся только доказать основную лемму для правил $\gamma$ и $\delta$.
    \end{itemize}
\end{frame}

\begin{frame}
    \frametitle{Основная лемма теоремы о корректности}
    \begin{itemize}
        \item Ветвь $\mathcal{B}$ выполнима, т.е. есть модель $M$, в которой все уравнения из $\mathcal{B}$ истинны. Параметры имеют значения в этой модели, как константы.
        \item Случай $\gamma$ с термом $t$: Рассмотрим $\forall v ~ A(v)=1$. По предположению, она истинна в $M$. Значит, $A(t)=1$ тоже истинно в $M$. Для $\exists v ~ A(v)=0$ аналогично.
        \pause
        \item Случай $\delta$ с параметром $a_i$: Рассмотрим $\exists v ~ A(v)=1$. По предположению, она истинна в $M$. Значит, есть $e \in \bar{M}$, для которого $A(e) = 1$ \pause (строго говоря, $\sigma_{v \mapsto e}(A(v)) = 1$).
        \item Зададим $M'$, которая отличается от $M$ только тем, что параметр $a_i$ имеет значение $e$. Тогда в $M'$: \begin{itemize}
            \item $A(a_i) = A(e) = 1$; 
            \item остальные уравнения из $\mathcal{B}$ истинны, так как \pause не содержат $a_i$.
        \end{itemize}
        Значит, все уравнения новой ветви истинны в $M'$.
    \end{itemize}
\end{frame}

\begin{frame}
    \frametitle{Множества и лемма Хинтикки}
    \begin{itemize}
        \item Перейдём к теореме о полноте.
        \item Для упрощения ограничимся случаем, когда в сигнатуре есть только предикатные символы, нет константных и функциональных.
        \pause
        \item Множество уравнений $H$ "--- \emph{множество Хинтикки}, если есть множество параметров $U$ такое, что
        \begin{enumerate}
            \item $P(a_{i_1},\ldots,a_{i_k})=1 \notin H \lor P(a_{i_1},\ldots,a_{i_k})=0 \notin H$ для любого предикатного символа $P$ и параметров $a_{i_1},\ldots,a_{i_k}$ из $U$.
            \item $\alpha \in H \Rightarrow \alpha_1 \in H \land \alpha_2 \in H$.
            \item $\beta \in H \Rightarrow \beta_1 \in H \lor \beta_2 \in H$.
            \item $\gamma \in H \Rightarrow \forall a_i \in U ~ \gamma(a_i) \in H$.
            \item $\delta \in H \Rightarrow \exists a_i \in U ~ \delta(a_i) \in H$.
        \end{enumerate}
        \pause
        \item Лемма Хинтикки для логики первого порядка: любое множество Хинтикки имеет модель, носителем которой является $U$.
    \end{itemize}
\end{frame}

\begin{frame}
    \frametitle{Лемма Хинтикки (доказательство)}
    \begin{itemize}
        \item Как и для логики высказываний, задаём
        \[ P^M(a_{i_1},\ldots,a_{i_k}) = \left\{ \begin{array}{l}
            1, \text{ если } P(a_{i_1},\ldots,a_{i_k})=1 \in H \\
            0, \text{ если } P(a_{i_1},\ldots,a_{i_k})=0 \in H \\
            0 \text{ иначе} 
        \end{array} \right. \]
        \item Атомарные уравнения из $H$ истинны в $M$ по определению. 
        \pause
        \item Индукцией по сложности покажем, что все остальные тоже.
        \item $\alpha$ и $\beta$ как в логике высказываний.
        \pause
        \item Пусть $\gamma \in H$ имеет вид $\forall v ~ A(v)=1$. Тогда все $A(a_i)=1 \in H$. Их сложность меньше, и они истинны в $M$ по предположению индукции. Значит, $\forall v ~ A(v)=1$ истинно в $M$. 
        \pause
        \item Для $\exists v ~ A(v)=0$ аналогично.
        \pause
        \item Для $\delta$ тоже разбираем отдельно $\exists v ~ A(v)=1$ и $\forall v ~ A(v)=0$.
    \end{itemize}
\end{frame}

\begin{frame}
    \frametitle{Систематические деревья}
    \begin{itemize}
        \item В отличие от логики высказываний, порядок раскрытия уравнений в логике предикатов важен.
        \item То есть может оказаться, что в зависимости от этого порядка ветвь закроется или нет.
        \item Нужен способ гарантировать, что если противоречие есть, мы его найдём
        \item А если нет, то получим множество Хинтикки.
        \item Этот способ "--- систематические деревья.
        \pause
        \item На каждом шаге раскрываем самое верхнее нераскрытое уравнение. Если их несколько на одном уровне, то левое из них.
        \item Уравнения типа $\gamma$ раскрываем сначала с параметром $a_1$, и добавляем под $\gamma(a_1)$ ещё одну копию $\gamma$.
        \item Когда до неё дойдём, раскроем с $a_2$ и добавим ещё копии\ldots
    \end{itemize}
\end{frame}

\begin{frame}
    \frametitle{Систематические деревья (2)}
    \begin{itemize}
        \item Систематическое дерево \emph{закончено} в любом из следующих случаев: 
        \begin{enumerate}
            \item оно закрыто;
            \item в нём не осталось неразобранных уравнений;
            \item оно бесконечно.
        \end{enumerate}
        \item Открытая ветвь в законченном систематическом дереве является множеством Хинтикки.
        \item Доказательство: если ветвь конечна, это следует сразу из определения. 
        \item А если она бесконечна, то в систематической процедуре было сделано бесконечно много шагов.
        \item Так как число уравнений на каждом уровне конечно, то все уравнения до любого фиксированного уровня $n$ будут разобраны за конечное число шагов.
        \item Значит, все уравнения бесконечной ветви разобраны. В том числе типа $\gamma$ "--- со всеми параметрами.
    \end{itemize}
\end{frame}

\begin{frame}
    \frametitle{Теорема о полноте}
    \begin{itemize}
        \item Теорема: если $A$ тождественно истинна, то законченное систематическое дерево истинности для $A=0$ закрыто.
        \pause
        \item Доказательство: если оно открыто, то его открытая ветвь "--- множество Хинтикки, содержащее $A=0$. 
        \item По лемме Хинтикки эта ветвь имеет модель.
        \item В этой модели $A=0$.
        \item Значит, $A$ не тождественно истинна.
    \end{itemize}
\end{frame}

\begin{frame}
    \frametitle{Счётные модели}
    \begin{itemize}
        \item Существенно, что построенная модель для множества Хинтикки конечна или \emph{счётна}, то есть её элементы можно перенумеровать $\{a_1,a_2,\ldots\}$.\footnote{Больше о счётных и несчётных множествах в следующем разделе.}
        \item Пусть формула $A$ выполнима (то есть имеет какую-то модель). 
        \item Тогда $\neg A$ не тождественно истинна.
        \item Значит, систематическое дерево для $\neg A = 0$ или для $A = 1$ не может быть закрыто.
        \item Значит, $A$ имеет конечную или счётную модель.
    \end{itemize}
\end{frame}

\begin{frame}
    \frametitle{Теорема Лёвенгейма-Сколема}
    \begin{itemize}
        \item Можно рассматривать деревья не для одного уравнения, а для множества, в том числе счётного.
        \item Позже увидим, что любое бесконечное множество уравнений логики предикатов счётно.
        \item Если множество счётное, то мы не можем начать с ветви, содержащей все уравнения.
        \item Вместо этого разрешаем кроме обычных правил расширения дерева добавлять уравнения из множества в конец открытой ветви.
        \item В построении систематического дерева будет изменение: после шага $n$ добавим в конец всех открытых ветвей уравнение $A_n$.
        \item Доказательства выше работают по-прежнему.
        \item Получается теорема Лёвенгейма-Сколема: если счётное множество уравнений выполнимо, то оно выполнимо в конечной или счётной модели.
    \end{itemize}
\end{frame}

\begin{frame}
    \frametitle{Теорема о компактности}
    \begin{itemize}
        \item С другой стороны, если систематическое дерево для бесконечного множества уравнений закроется, то это случится после конечного количества шагов.
        \item При этом может быть использовано только конечное число уравнений исходного множества.
        \item Теорема о компактности: если множество уравнений невыполнимо, то у него есть конечное невыполнимое подмножество.
    \end{itemize}
\end{frame}

\begin{frame}
\frametitle{Последние замечания}
\begin{itemize}
    \item Есть варианты правил деревьев, которые позволят найти конечный контрпример в последнем примере, но их сложнее объяснить и понять.
    \item Краткое изложение \href{https://en.wikipedia.org/wiki/Method_of_analytic_tableaux\#First-order_tableau_with_unification}{в английской Википедии}. \pause
    \item[]
    \item Пока мы не можем это доказать, но логика предикатов неразрешима: нет алгоритма, который для любой формулы скажет, тождественно истинна ли она.
    \item Значит, в любой системе доказательств надо что-то придумывать (для некоторых формул), чисто механических действий недостаточно.
\end{itemize}
\end{frame}


\end{document}