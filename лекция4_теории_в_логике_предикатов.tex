\documentclass[10pt]{beamer}
\usepackage[utf8]{inputenc}
\usepackage[T1,T2A]{fontenc}
\usepackage[russian]{babel}
\usepackage{color}
\usepackage{calc}
\usepackage{graphicx}
\usepackage{epstopdf}
\usepackage{hyperref}
\hypersetup{unicode,colorlinks}
\usepackage{csquotes}
\usepackage{upquote}
\usepackage{cprotect}
\usetheme[progressbar=head,numbering=fraction,block=fill]{metropolis}
\usepackage{dejavu}
\usepackage{etoolbox}
\usepackage{bm}
\usepackage[ND]{prftree}
\usepackage[tableaux]{prooftrees}
\usepackage{mathtools} % for bigtimes
\usepackage{diagbox}
\usepackage{verbatimbox}
\usepackage{fitch}
\usepackage{enumitem}
%\usepackage[defaultsans]{droidsans}
%\usepackage[defaultmono]{droidmono}
%\makeatletter
%\input droid/t1fdm.fd
%\makeatother

\makeatletter
\DeclareRobustCommand*{\bfseries}{%
    \not@math@alphabet\bfseries\mathbf
    \fontseries\bfdefault\selectfont
    \boldmath
}
\makeatother

\makeatletter
\chardef\straightquote@code=\catcode`'
\chardef\backquote@code=\catcode``
\catcode`'=\active \catcode``=\active
\patchcmd{\@noligs}
{\textasciigrave}
{\fixedtextasciigrave}
{}{}
\newcommand{\fixedtextasciigrave}{%
    \makebox[.5em]{\fontencoding{TS1}\fontfamily{fvs}\selectfont\textasciigrave}% Vera Sans
}
\catcode\lq\'=\straightquote@code
\catcode\lq\`=\backquote@code
\makeatletter

\setbeamercovered{again covered={\opaqueness<1->{25}}}

\forestset{close with=\ensuremath{\bigtimes}}

\newcommand{\customframefont}[1]{
    \setbeamertemplate{itemize/enumerate body begin}{#1}
    \setbeamertemplate{itemize/enumerate subbody begin}{#1}
}

\NewEnviron{framefont}[1]{
    \customframefont{#1} % for itemize/enumerate
    {#1 % For the text outside itemize/enumerate
        \BODY
    }
    \customframefont{\normalsize}
}


\makeatletter
\DeclareRobustCommand{\rvdots}{%
    \vphantom{\int\limits^x}\smash[t]{\vdots}
}
\makeatother

\newcommand{\N}{\mathbb{N}}
\newcommand{\Z}{\mathbb{Z}}
\newcommand{\Q}{\mathbb{Q}}
\newcommand{\R}{\mathbb{R}}
\newcommand{\pow}[1]{\mathcal{P}({#1})}
\newcommand{\continuum}{\mathfrak{c}}

\author{Алексей Романов}
\subtitle{Математическая логика и теория алгоритмов}
%\logo{}
\institute{МИЭТ}
\subject{Математическая логика и теория алгоритмов}
%\setbeamercovered{transparent}
%\setbeamertemplate{navigation symbols}{}


\title{Логика предикатов\\Теории}
\date{\today}

\begin{document}
\begin{frame}[plain]
\maketitle
\end{frame}

\begin{frame}
\frametitle{Классы моделей}
\begin{itemize}
    \item Практически всегда интересны не все модели данной сигнатуры, а какой-то их класс (или одна модель).
    \item Примеры больших классов моделей: графы, линейные пространства и т.д.
    \item Такие классы часто задаются наборами аксиом.
    \item Натуральные числа (в логике обычно включая $0$) с операциями сложения и умножения "--- \enquote{стандартная модель} для сигнатуры арифметики.
    \item Действительные "--- для мат.анализа.
    \item Для конкретных моделей можно поставить вопрос: есть ли набор аксиом, полностью описывающих эту модель?
\end{itemize}
\end{frame}

\begin{frame}
    \frametitle{Теории}
    \begin{itemize}
        \item \emph{Теория} $T$ "--- множество замкнутых формул какой-то сигнатуры $\sigma_T$, называемых \emph{аксиомами} $T$. 
        \item Дальше все модели и формулы "--- сигнатуры $\sigma_T$.
        \pause
        \item Модель $M$ называется \emph{моделью $T$}, если в ней истинны все аксиомы $T$. Пишется $M \vDash T$.
        \pause
        \item Замкнутая формула $A$ называется \emph{теоремой $T$} или \emph{выводимой из $T$}, если её можно доказать из аксиом $T$. Пишется $T \vdash A$.
        \pause
        \item Переформулировка теоремы о корректности: $T \vdash A$ $\Rightarrow$ $A$ истинна во всех моделях $T$.
        \item Переформулировка теоремы о полноте: $A$ истинна во всех моделях $T$ $\Rightarrow T \vdash A$.
    \end{itemize}
\end{frame}

\begin{frame}
    \frametitle{Некоторые простые свойства выводимости}
    \begin{itemize}
        \item Рефлексивность: каждая аксиома $T$ является её теоремой.
        \pause
        \item Транзитивность: если все аксиомы $T_1$ "--- теоремы $T_2$, то все теоремы $T_1$ "--- теоремы $T_2$.
        \pause
        \item Монотонность: если $T_1 \subset T_2$, то все теоремы $T_1$ "--- теоремы $T_2$.
        \pause
        \item Теорема о дедукции: если $T \cup \{A\} \vdash B$, то $T \vdash A \to B$.
        \pause
        \item Заметили ли, что монотонность следует из рефлексивности и транзитивности?
    \end{itemize}
\end{frame}

\begin{frame}
    \frametitle{Противоречивость теории}
    \begin{itemize}
        \item Теория называется \emph{противоречивой}, если в ней выводимы одновременно какая-то формула $A$ и $\neg A$.
        \item В этом случае в ней выводимы все формулы (её сигнатуры).
        \item Противоречивая теория не имеет моделей.
        \item А непротиворечивая имеет.
        \item Переформулировка теоремы Лёвенгейма-Сколема: непротиворечивая теория имеет конечную или счётную модель.
    \end{itemize}
\end{frame}

\begin{frame}
    \frametitle{Аксиомы равенства}
    \begin{itemize}
        \item Пусть в сигнатуре есть предикат равенства $=$.
        \item Тогда \emph{нормальная модель} "--- такая, в которой он интерпретируется как обычное равенство. 
        \item Какие формулы истинны на всех нормальных моделях?
        \pause
        \item $\forall x ~ x=x$
        \item $\forall x \forall y ~ x=y \to y=x$
        \item $\forall x \forall y \forall z ~ x=y \land y=z \to x=z$
        \pause
        \item Для каждого $n$-местного функционального символа $f$: \pause $\forall x_1 \ldots \forall x_n \forall y_1 \ldots \forall y_n ~ x_1 = y_1 \land \ldots \land x_n = y_n \to f(x_1,\ldots,x_n) = f(y_1,\ldots,y_n)$.
        \pause
        \item Для каждого $n$-местного предикатного символа $P$: \pause $\forall x_1 \ldots \forall x_n \forall y_1 \ldots \forall y_n ~ x_1 = y_1 \land \ldots \land x_n = y_n \to (P(x_1,\ldots,x_n) \to P(y_1,\ldots,y_n))$.
        \pause
        \item Всё это \emph{аксиомы равенства}.
    \end{itemize}
\end{frame}

\begin{frame}
    \frametitle{Теорема о полноте для нормальных моделей}
    \begin{itemize}
        \item Теорема о полноте для нормальных моделей: если в непротиворечивой теории выводимы все аксиомы равенства для её сигнатуры, то она имеет нормальную модель.
        \item Идея доказательства: есть модель по обычной теореме о полноте, но не обязательно нормальная. 
        \item Интерпретация равенства в этой модели "--- \pause отношение эквивалентности. 
        \item Можем взять множество классов эквивалентности по этому отношению (фактор-множество). 
        \item Значения функций и предикатов на фактор-множестве задаются однозначно \pause (каким именно образом?).
        \item В теореме Лёвенгейма-Сколема и в теореме о компактности тоже можно потребовать, чтобы модель была нормальной.
    \end{itemize}
\end{frame}

\begin{frame}
    \frametitle{Пример: теория графов}
    \begin{itemize}
        \item Мы можем описать графы несколькими возможными сигнатурами. 
        \item Вариант 1:
        \begin{itemize}
            \item 1 сорт объектов: вершины.
            \item Предикат $E(x,y)$: вершины $x$ и $y$ смежны.
            \item Аксиомы (кроме аксиом равенства):
            \pause
            \begin{enumerate}
                \item $\forall x ~ \neg E(x,x)$
                \item $\forall x \forall y ~ E(x,y) \to E(y,x)$
            \end{enumerate}
        \end{itemize}
        \pause
        \item Вариант 2:
        \begin{itemize}
            \item 2 сорта объектов: вершины $V$ и рёбра $E$.
            \item Предикат $I(v,e)$: вершина $v$ инцидентна ребру $e$.
            \item Аксиомы:
            \pause
            \begin{enumerate}
                \item $\forall e \nolinebreak : \nolinebreak E ~ \exists v_1 \nolinebreak : \nolinebreak V ~ \exists v_2 \nolinebreak : \nolinebreak V ~ I(v_1,e) \land I(v_2,e) \land v_1 \neq v_2 \land \linebreak \forall v_3 \nolinebreak : \nolinebreak V (I(v_3,e) \to v_3 = v_1 \lor v_3 = v_2)$
                \item $\forall v_1 \nolinebreak : \nolinebreak V ~ \forall v_2 \nolinebreak : \nolinebreak V ~  \forall e_1 \nolinebreak : \nolinebreak E ~ \forall e_2 \nolinebreak : \nolinebreak E ~ \linebreak I(v_1,e_1) \land I(v_2,e_1) \land I(v_1,e_2) \land I(v_2,e_2) \to e_1 = e_2$
            \end{enumerate}
        \end{itemize}
    \pause
    \item Что изменится для обобщённых графов? Для ориентированных?
    \pause
    \item Какая теория выразительнее?
    \end{itemize}
\end{frame}

\begin{frame}
    \frametitle{Формальные арифметики}
    \begin{itemize}
        \item Рассмотрим теории первого порядка в языке арифметики. Обычно включают:
        \begin{itemize}
            \item константу $0$;
            \item предикат равенства;
            \item функции $S$ (следующее число), $+$ и $\cdot$.
        \end{itemize}
        \item Остальные константы, предикаты и функции выражаются через них.
        \item Сегодня рассмотрим в первую очередь арифметику Пеано.
    \end{itemize}
\end{frame}

\begin{frame}
    \frametitle{Арифметика Пеано}
    \begin{itemize}
        \item $\forall x ~ S(x) \neq 0$
        \item $\forall x \forall y ~ S(x) = S(y) \to x = y$
        \item $\forall x ~ x + 0 = x$
        \item $\forall x ~ x + S(y) = S(x+y)$
        \item $\forall x ~ x \cdot 0 = 0$
        \item $\forall x ~ x \cdot S(y) = x \cdot y + x$
        \item Схема индукции: для каждой формулы $A(x,y_1,\ldots,y_n)$ ($x,y_1,\ldots,y_n$ "--- все её свободные переменные): \\
        $\forall y_1 \ldots \forall y_n ~ (A(0,y_1,\ldots,y_n) \land \linebreak \forall x ~ (A(x,y_1,\ldots,y_n) \to A(S(x),y_1,\ldots,y_n)) \to \linebreak \forall x ~ A(x,y_1,\ldots,y_n))$
        \item[]
        \item В этой теории можно доказать все стандартные арифметические утверждения.
    \end{itemize}
\end{frame}

\begin{frame}
    \frametitle{Нестандартные модели арифметики}
    \begin{itemize}
        \item Пусть $T$ "--- арифметическая теория, истинная на $\mathbb{N}$ (например, арифметика Пеано). Тогда у неё есть счётная модель, неизоморфная $\mathbb{N}$.
        \item Доказательство: добавим к сигнатуре арифметики константу $c$, а к $T$ аксиомы $c > 0, c > 1, c > 2, \ldots$. Получим теорию $T'$. \pause 
        \item Любое конечное подмножество $T'$ имеет модель. 
        \item Значит, по теореме о компактности $T'$ имеет конечную или счётную модель. 
        \pause 
        \item Конечной модели у $T'$ быть не может, значит, модель счётная. Но она будет и моделью $T$.
    \end{itemize}
\end{frame}

\begin{frame}
    \frametitle{Полные теории}
    \begin{itemize}
        \item Непротиворечивая теория $T$ называется \emph{полной}, если для каждой замкнутой формулы $A$ (её сигнатуры) либо $T \vdash A$, либо $T \vdash \neg A$.
        \item Пример: для любой модели $M$ можно рассмотреть теорию $Th(M)$, аксиомы которой "--- все формулы, истинные в $M$. Она \pause полна: если $M \vDash A$, то $Th(M) \vdash A$, а иначе $Th(M) \vdash \neg A$.
        \item Полна ли теория графов? \pause Нет. Например, формула $\exists x \exists y ~ E(x,y)$ истинна в одних графах и ложна в других. Значит, ни она, ни её отрицание "--- не теоремы.
        \pause
        \item Теоремы Гёделя о неполноте говорят, что никакая теория с разрешимым множеством аксиом, позволяющая доказать арифметику Робинсона, не может быть полна. Подробнее о них в конце курса, если успеем.
    \end{itemize}
\end{frame}

\begin{frame}
    \frametitle{Разрешимые теории}
    \begin{itemize}
        \item Теория $T$ называется \emph{разрешимой}, если существует алгоритм, который по замкнутой формуле $A$ определяет, является ли она теоремой $T$.
        \item В прошлый раз упоминалось, что чистая логика предикатов (т.е. пустая теория) неразрешима.
        \item Как простой пример разрешимой теории приведём $Th(M)$ для конечной модели $M$.
        \item Ещё примеры неразрешимых теорий: теория полугрупп, $Th(\mathbb{N})$ (в сигнатуре $+,\cdot,=$), арифметика Робинсона (и любое её непротиворечивое расширение).
        \item Примеры разрешимых теорий: теория равенства (в сигнатуре только $=$, аксиомы равенства), $Th(\mathbb{C})$ в сигнатуре $+,\cdot,=$, арифметика Пресбургера (арифметика Пеано без умножения).
    \end{itemize}
\end{frame}

\end{document}