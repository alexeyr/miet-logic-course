\documentclass[10pt]{beamer}
\usepackage[utf8]{inputenc}
\usepackage[T1,T2A]{fontenc}
\usepackage[russian]{babel}
\usepackage{color}
\usepackage{calc}
\usepackage{graphicx}
\usepackage{epstopdf}
\usepackage{hyperref}
\hypersetup{unicode,colorlinks}
\usepackage{csquotes}
\usepackage{upquote}
\usepackage{cprotect}
\usetheme[progressbar=head,numbering=fraction,block=fill]{metropolis}
\usepackage{dejavu}
\usepackage{etoolbox}
\usepackage{bm}
\usepackage[ND]{prftree}
\usepackage[tableaux]{prooftrees}
\usepackage{mathtools} % for bigtimes
\usepackage{diagbox}
\usepackage{verbatimbox}
\usepackage{fitch}
\usepackage{enumitem}
%\usepackage[defaultsans]{droidsans}
%\usepackage[defaultmono]{droidmono}
%\makeatletter
%\input droid/t1fdm.fd
%\makeatother

\makeatletter
\DeclareRobustCommand*{\bfseries}{%
    \not@math@alphabet\bfseries\mathbf
    \fontseries\bfdefault\selectfont
    \boldmath
}
\makeatother

\makeatletter
\chardef\straightquote@code=\catcode`'
\chardef\backquote@code=\catcode``
\catcode`'=\active \catcode``=\active
\patchcmd{\@noligs}
{\textasciigrave}
{\fixedtextasciigrave}
{}{}
\newcommand{\fixedtextasciigrave}{%
    \makebox[.5em]{\fontencoding{TS1}\fontfamily{fvs}\selectfont\textasciigrave}% Vera Sans
}
\catcode\lq\'=\straightquote@code
\catcode\lq\`=\backquote@code
\makeatletter

\setbeamercovered{again covered={\opaqueness<1->{25}}}

\forestset{close with=\ensuremath{\bigtimes}}

\newcommand{\customframefont}[1]{
    \setbeamertemplate{itemize/enumerate body begin}{#1}
    \setbeamertemplate{itemize/enumerate subbody begin}{#1}
}

\NewEnviron{framefont}[1]{
    \customframefont{#1} % for itemize/enumerate
    {#1 % For the text outside itemize/enumerate
        \BODY
    }
    \customframefont{\normalsize}
}


\makeatletter
\DeclareRobustCommand{\rvdots}{%
    \vphantom{\int\limits^x}\smash[t]{\vdots}
}
\makeatother

\newcommand{\N}{\mathbb{N}}
\newcommand{\Z}{\mathbb{Z}}
\newcommand{\Q}{\mathbb{Q}}
\newcommand{\R}{\mathbb{R}}
\newcommand{\pow}[1]{\mathcal{P}({#1})}
\newcommand{\continuum}{\mathfrak{c}}

\author{Алексей Романов}
\subtitle{Математическая логика и теория алгоритмов}
%\logo{}
\institute{МИЭТ}
\subject{Математическая логика и теория алгоритмов}
%\setbeamercovered{transparent}
%\setbeamertemplate{navigation symbols}{}


\title{Ординалы}
\date{\today}

\begin{document}
\begin{frame}[plain]
    \maketitle
\end{frame}

\begin{frame}
    \frametitle{Ординалы как множества}
    \begin{itemize}
        \item $0 = \varnothing$
        \item $\forall \alpha : Ord ~ \alpha + 1 = \alpha \cup \{\alpha\}$
        \item $\forall A \subset Ord ~ \sup A = \bigcup A$ 
        \pause
        \item Например, $0 = \varnothing; 1 = \{0\} = \{\varnothing\}; 2 = \{0,1\} = \{\varnothing, \{\varnothing\}\}; \ldots$
        \item $\omega = \{0,1,2,\ldots\}$
        \pause
        \item $\alpha < \beta \Leftrightarrow \alpha \subsetneq \beta \Leftrightarrow \alpha \in \beta$
        \item Всегда $\alpha = \{\beta \mid \beta < \alpha\}$.
        \pause
        \item $\omega = \sup \omega = \sup \{\alpha | \alpha < \omega\}$. Такой ординал называется предельным.
        \item Дальше $\omega + 1, \omega + 2, \ldots, \omega \cdot 2, \omega \cdot 2 + 1, \ldots, \omega \cdot 3, \ldots, \omega \cdot 4, \ldots, \omega^2, \ldots$
        \item $\omega \cdot 2 = \sup \{\omega + 1, \omega +2, \ldots\}$
        \item $\omega \cdot \omega = \sup \{\omega, \omega \cdot 2, \ldots\}$
    \end{itemize}
\end{frame}

\begin{frame}
    \frametitle{Арифметика ординалов}
    \begin{itemize}
        \item $\alpha + \beta$: порядок на $\alpha \cup \beta'$ (копия $\beta$), все элементы $\alpha$ меньше всех элементов $\beta'$.
        \pause
        \item $\alpha \cdot \beta$: лексикографический порядок на $\beta \times \alpha$ (сравнить первые элементы, если они равны, сравнить вторые).
        \pause
        \item Рекурсивные определения:
        \begin{itemize}
            \item $\alpha + 0 = \alpha$
            \item $\alpha + (\beta + 1) = (\alpha + \beta) + 1$
            \item $\alpha + \sup A = \sup \{\alpha + \beta \mid \beta : A\}$
            \pause
            \item[]
            \item $\alpha \cdot 0 = 0$
            \pause            
            \item $\alpha \cdot (\beta + 1) = \alpha \cdot \beta + \alpha$
            \pause
            \item $\alpha \cdot \sup A = \sup \{\alpha \cdot \beta \mid \beta : A\}$
            \pause
            \item[]
            \item $\alpha^0 = 1$
            \pause            
            \item $\alpha^{\beta + 1} = \alpha ^ \beta \cdot \alpha$
            \pause
            \item $\alpha^{\sup A} = \sup \{\alpha ^ \beta \mid \beta : A\}$
        \end{itemize}
    \end{itemize}
\end{frame}

\begin{frame}
    \frametitle{Примеры}
    \begin{itemize}
        \item Рекурсией и напрямую: $\omega + 2$ и $2 + \omega$, $\omega \cdot 2$ и $2 \cdot \omega$.
        \pause
        \item $(\omega+1)^2 \pause = \omega^2 + \omega + 1$
        \pause
        \item $2^\omega \pause = \omega$. Заметьте, что операции над ординалами не согласуются с операциями над мощностями, кроме конечных чисел.
        \pause
        \item[]
        \item Примеры выполняющихся свойств:
        \begin{itemize}
            \item $0 + \alpha = \alpha$
            \item $(\alpha + \beta) + \gamma = \alpha + (\beta + \gamma)$ (и для $\cdot$)
            \item $\alpha \cdot (\beta + \gamma) = \alpha \cdot \beta + \alpha \cdot \gamma$
            \item $\beta > \gamma \Rightarrow \alpha + \beta > \alpha + \gamma$
            \item $\beta > \gamma \Rightarrow \beta + \alpha \geq \gamma + \alpha$
            \item $\ldots$
        \end{itemize}
        \item Примеры невыполняющихся свойств:
        \begin{itemize}
            \item $\alpha + \beta = \beta + \alpha$ (и для $\cdot$)
            \item $(\beta + \gamma) \cdot \alpha = \beta \cdot \alpha + \gamma \cdot \alpha$
            \item $\beta > \gamma \Rightarrow \beta + \alpha > \gamma + \alpha$
            \item $\ldots$            
        \end{itemize}
    \end{itemize}
\end{frame}

\begin{frame}
    \frametitle{Нормальная форма Кантора}
    \begin{itemize}
        \item Нормальная форма Кантора: любое $\alpha$ единственным образом представляется как $\omega^{\beta_1} \cdot c_1 + \ldots + \omega^{\beta_k} \cdot c_k$, где:
        \begin{itemize}
            \item $k \in \mathbb{N}$
            \item $c_i \in \mathbb{N}$
            \item $\beta_1 > \beta_2 > \ldots > \beta_k \geq 0 \in Ord$
        \end{itemize}
        \pause
        \item $\varepsilon_0 = \sup \{\omega, \omega^\omega, \omega^{\omega^\omega}, \ldots\}$
        \item $\omega^{\varepsilon_0} = \pause \varepsilon_0$ (и это минимальный такой ординал).
        \item Дальше эта иерархия продолжается неограниченно.
    \end{itemize}
\end{frame}


\end{document}