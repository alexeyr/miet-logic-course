\documentclass[10pt]{beamer}
\usepackage[utf8]{inputenc}
\usepackage[T1,T2A]{fontenc}
\usepackage[russian]{babel}
\usepackage{color}
\usepackage{calc}
\usepackage{graphicx}
\usepackage{epstopdf}
\usepackage{hyperref}
\hypersetup{unicode,colorlinks}
\usepackage{csquotes}
\usepackage{upquote}
\usepackage{cprotect}
\usetheme[progressbar=head,numbering=fraction,block=fill]{metropolis}
\usepackage{dejavu}
\usepackage{etoolbox}
\usepackage{bm}
\usepackage[ND]{prftree}
\usepackage[tableaux]{prooftrees}
\usepackage{mathtools} % for bigtimes
\usepackage{diagbox}
\usepackage{verbatimbox}
\usepackage{fitch}
\usepackage{enumitem}
%\usepackage[defaultsans]{droidsans}
%\usepackage[defaultmono]{droidmono}
%\makeatletter
%\input droid/t1fdm.fd
%\makeatother

\makeatletter
\DeclareRobustCommand*{\bfseries}{%
    \not@math@alphabet\bfseries\mathbf
    \fontseries\bfdefault\selectfont
    \boldmath
}
\makeatother

\makeatletter
\chardef\straightquote@code=\catcode`'
\chardef\backquote@code=\catcode``
\catcode`'=\active \catcode``=\active
\patchcmd{\@noligs}
{\textasciigrave}
{\fixedtextasciigrave}
{}{}
\newcommand{\fixedtextasciigrave}{%
    \makebox[.5em]{\fontencoding{TS1}\fontfamily{fvs}\selectfont\textasciigrave}% Vera Sans
}
\catcode\lq\'=\straightquote@code
\catcode\lq\`=\backquote@code
\makeatletter

\setbeamercovered{again covered={\opaqueness<1->{25}}}

\forestset{close with=\ensuremath{\bigtimes}}

\newcommand{\customframefont}[1]{
    \setbeamertemplate{itemize/enumerate body begin}{#1}
    \setbeamertemplate{itemize/enumerate subbody begin}{#1}
}

\NewEnviron{framefont}[1]{
    \customframefont{#1} % for itemize/enumerate
    {#1 % For the text outside itemize/enumerate
        \BODY
    }
    \customframefont{\normalsize}
}


\makeatletter
\DeclareRobustCommand{\rvdots}{%
    \vphantom{\int\limits^x}\smash[t]{\vdots}
}
\makeatother

\newcommand{\N}{\mathbb{N}}
\newcommand{\Z}{\mathbb{Z}}
\newcommand{\Q}{\mathbb{Q}}
\newcommand{\R}{\mathbb{R}}
\newcommand{\pow}[1]{\mathcal{P}({#1})}
\newcommand{\continuum}{\mathfrak{c}}

\author{Алексей Романов}
\subtitle{Математическая логика и теория алгоритмов}
%\logo{}
\institute{МИЭТ}
\subject{Математическая логика и теория алгоритмов}
%\setbeamercovered{transparent}
%\setbeamertemplate{navigation symbols}{}


\title{Деревья истинности (семантические таблицы)}
\date{\today}

\begin{document}
\begin{frame}[plain]
\maketitle
\end{frame}

\begin{frame}
\frametitle{Системы доказательств в логике высказываний}
\begin{itemize}
    \item Мы знаем две системы доказательств:
    \pause
    \item Таблицы истинности
    \item[] Проблемы:
    \pause
    \begin{itemize}
        \item Размер (удваивается при добавлении переменной)
        \item Не обобщается на логику предикатов
    \end{itemize}
    \pause
    \item Натуральная дедукция / исчисление секвенций
    \item[] Проблемы:
    \pause
    \begin{itemize}
        \item Нужно думать! (Выбирать правила, формулы для \(E\neg\) и т.д.)
        \item Позволяет доказать тождественно истинные формулы, но не опровергнуть не тождественно истинные
    \end{itemize}
    \item[]
    \item Сегодня: деревья истинности (семантические таблицы)
    \item Не имеют ни одной из этих проблем (разве что размер для некоторых задач). Но есть другие!
\end{itemize}
\end{frame}

\begin{frame}
\frametitle{Пример дерева истинности}
\begin{itemize}
    \item Начнём с примера. Нужно доказать $p \to (q \to p \land q)$:
    \item[] 
\only<1>{
    \begin{tableau}{
    }
        [{p \to (q \to p \land q) = 0}, just=Дано]
    \end{tableau}
}
\only<2>{
    \begin{tableau}{
        }
        [{p \to (q \to p \land q) = 0}, just=Дано, checked
        [{p = 1}, just=1
        [{q \to p \land q = 0}, just = 1
        ]]]
    \end{tableau}
}
\only<3>{
    \begin{tableau}{
        }
        [{p \to (q \to p \land q) = 0}, just=Дано, checked
        [{p = 1}, just=1, checked
        [{q \to p \land q = 0}, just = 1
        ]]]
    \end{tableau}
}
\only<4>{
    \begin{tableau}{
        }
        [{p \to (q \to p \land q) = 0}, just=Дано, checked
        [{p = 1}, just=1, checked
        [{q \to p \land q = 0}, just = 1, checked
        [{q = 1}, checked, just=3
        [{p \land q = 0}, just=3
        ]]]]]
    \end{tableau}
}
\only<5>{
    \begin{tableau}{
        }
        [{p \to (q \to p \land q) = 0}, just=Дано, checked
        [{p = 1}, just=1, checked
        [{q \to p \land q = 0}, just = 1, checked
        [{q = 1}, checked, just=3
        [{p \land q = 0}, just=3
        [{p = 0}, just=5] [{q = 0}, just=5]
        ]]]]]
    \end{tableau}
}
\only<6>{
    \begin{tableau}{
        }
        [{p \to (q \to p \land q) = 0}, just=Дано, checked
        [{p = 1}, just=1, checked
        [{q \to p \land q = 0}, just = 1, checked
        [{q = 1}, checked, just=3
        [{p \land q = 0}, just=3
        [{p = 0}, just=5, close={2,6}] [{q = 0}, just=5, close={4,6}]
        ]]]]]
    \end{tableau}
}
\onslide<7->{
    \item Пусть $p \to (q \to p \land q) = 0$. 
\item Тогда $p = 1$ и $q \to p \land q = 0$.
\item Тогда $q = 1$ и $p \land q = 0$.
\item Отсюда $p = 0$ или $q = 0$. Но оба невозможны!
\item Значит,  $p \to (q \to p \land q) = 0$ невозможно.
\item $p \to (q \to p \land q)$ тождественно истинно!
}

\end{itemize}
\end{frame}

\begin{frame}
\frametitle{Правила деревьев истинности}
\begin{itemize}
    \item  Формулы со знаком ($A=0$ или $A=1$) делятся на 4 типа: 
    \begin{itemize}
        \item Атомарные (слева переменная).
        \item Слева отрицание.
        \item $\alpha$ ведут себя как \enquote{и}: $\alpha \Leftrightarrow \alpha_1 \land \alpha_2$.
        \item $\beta$ ведут себя как \enquote{или}: $\beta \Leftrightarrow \beta_1 \lor \beta_2$.
    \end{itemize}
    \pause
    \item[]
    \item Правила:
\begin{columns}
    \begin{column}{0.33\textwidth}
        \begin{center}
            \begin{tableau}{not line numbering, single branches}
                [{\neg A = 1(0)} [{~A = 0(1)}]]
            \end{tableau}
        \end{center}
    \end{column}
    \begin{column}{0.33\textwidth}
        \begin{center}
            \begin{tableau}{not line numbering, single branches}
                [{\alpha} [{\displaystyle \alpha_1 \atop \displaystyle \alpha_2}]]
            \end{tableau}
        \end{center}
    \end{column}
    \begin{column}{0.33\textwidth}
        \begin{center}
            \begin{tableau}{not line numbering}
                [\beta [\beta_1] [\beta_2]]
            \end{tableau}
        \end{center}
    \end{column}
\end{columns}
    \vspace{1em}
    \begin{tabular}{|c | c | c || c | c | c |}
        \hline
        $\alpha$ & $\alpha_1$ & $\alpha_2$ & $\beta$ & $\beta_1$ & $\beta_2$ \\
        \hline
        $A \land B = 1$ & $A = 1$ & $B = 1$ & $A \land B = 0$ & $A = 0$ & $B = 0$ \\
        $A \lor B = 0$ & $A = 0$ & $B = 0$ & $A \lor B = 1$ & $A = 1$ & $B = 1$ \\
        $A \to B = 0$ & $A = 1$ & $B = 0$ & $A \to B = 1$ & $A = 0$ & $B = 1$ \\
        \hline
    \end{tabular}
    \pause
\end{itemize}
\end{frame}

\begin{frame}
\frametitle{Построение дерева истинности}
\begin{itemize}
    \item Идея в поиске контрпримера для формулы (или секвенции), которую хотим доказать.
    \item Для доказательства формулы $A$ мы начинаем с уравнения $A=0$.
    \item Для секвенции $A_1,\, A_2,\,\ldots \vdash B$ начинаем с \pause $A_1=1,\, A_2=1,\, \ldots,\, B=0$ (пишем их в столбик).
    \item Для эквивалентности $A \equiv B$ строим 2 дерева: \pause $A=1,\, B=0$ и $A=0,\, B=1$.
    \pause
    \item[]
    \item На каждом шаге:
    \begin{itemize}
        \item Выбираем неразобранное уравнение (не отмечено \checkmark).
        \item Применяем правила с предыдущего слайда.
        \item Результаты дописываются в конец всех открытых ветвей под ним.
        \item Отмечаем как разобранное.
    \end{itemize}
\end{itemize}
\end{frame}

\begin{frame}
\frametitle{Построение дерева истинности (2)}
\begin{itemize}
    \item Ветвь, в которой одновременно есть $A=1$ и $A=0$, называется закрытой.
    \item В ней можно дальше ничего не писать.
    \item[]
    \item Если все ветви закрылись, то уравнения, с которых начали, не имеют решений! 
    \item Значит, исходная формула/секвенция не имеет контрпримера и она тождественно истинна.
    \item[]
    \pause
    \item Если какая-то ветвь закончилась (нет неразобранных строк), но не закрылась, то мы нашли контрпример к исходной формуле и она \emph{не} тождественно истинна.
    \item[]
    \pause
    \item Порядок применения влияет только на размер дерева, но не на результат.
    \item Выгоднее сначала применять правила без ветвления.
\end{itemize}
\end{frame}

\begin{frame}
\frametitle{Пример незакрытого дерева}
\begin{itemize}
    \item Проверим $(p \to r) \lor (q \to r) \to (p \lor q \to r)$:
    \begin{tableau}{}
        [{(p \to r) \lor (q \to r) \to (p \lor q \to r) = 0}, checked, just=Дано
        [{(p \to r) \lor (q \to r) = 1}, checked, just=1
        [{p \lor q \to r = 0}, checked, just=1
        [{p \lor q = 1}, checked, just=3
        [{r = 0}, checked, just=3
        [{p \to r = 1}, checked, just=2
        [{p = 1}, checked, just=4
        [{p=0}, checked, just=6, close={7,8}]
        [{r=1}, checked, just=6, close={5,8}]
        ]
        [{q = 1}, checked, just=4
        [\vdots]
        ]
        ]
        [{q \to r = 1}, checked, just=2
        [{p = 1}, checked, just=4
        [{q=0}, checked, just=6]
        [{r=1}, checked, just=6, close={5,8}]
        ]
        [{q = 1}, checked, just=4
        [\vdots]
        ]
        ]
        ]]]]]
    \end{tableau}
    \item В законченной ветви $p=1,\,q=0,\,r=0$. На этом наборе строка 1 выполняется, значит, \pause формула не тождественно истинна.
\end{itemize}
\end{frame}

\begin{frame}
\frametitle{Ещё один пример}
\begin{itemize}
    \item Для $p \vee (q \vee \lnot r), p \to \lnot r, q \to \lnot r \vdash \lnot r$:
    \begin{tableau}{}
        [{p \lor (q \lor \lnot r) = 1}, just=Дано, checked
        [{p \to \lnot r = 1}, just=Дано, checked
        [{q \to \lnot r = 1}, just=Дано, checked,name=last premise
        [{\lnot r = 0}, just=Дано,name=not conc
        [{p = 1}, just={1}
        [{p = 0}, close={:!u,!c}]
        [{\lnot r = 1}, close={:not conc,!c},just={ 2}]]
        [{q \vee \lnot r = 1}
        [{q = 1}, move by=1
        [{q = 0}, close={:!u,!c}]
        [{\lnot r = 1}, close={:not conc,!c},just={3}]]
        [{\lnot r = 1}, close={:not conc,!c},move by=1, just={7}]]]]]]
    \end{tableau}
    \item Дерево закрыто, значит, \pause контрпримера нет и секвенция тождественно истинна.
\end{itemize}
\end{frame}

\begin{frame}
\frametitle{Домашнее задание}
\begin{enumerate}
    \item[] С помощью деревьев докажите или опровергните:
    \item $p \to q \equiv \neg p \lor q$
    \item $(p \land q) \lor r \equiv (p \lor r) \lor (q \lor r)$
    \item $\neg (p \land q) \equiv \neg p \land \neg q$
    \item $p \land q \to r \equiv (p \to r) \lor (q \to r)$
\end{enumerate}
\end{frame}


\end{document}