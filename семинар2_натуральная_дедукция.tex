\documentclass[10pt]{beamer}
\usepackage[utf8]{inputenc}
\usepackage[T1,T2A]{fontenc}
\usepackage[russian]{babel}
\usepackage{color}
\usepackage{calc}
\usepackage{graphicx}
\usepackage{epstopdf}
\usepackage{hyperref}
\hypersetup{unicode,colorlinks}
\usepackage{csquotes}
\usepackage{upquote}
\usepackage{cprotect}
\usetheme[progressbar=head,numbering=fraction,block=fill]{metropolis}
\usepackage{dejavu}
\usepackage{etoolbox}
\usepackage{bm}
\usepackage[ND]{prftree}
\usepackage[tableaux]{prooftrees}
\usepackage{mathtools} % for bigtimes
\usepackage{diagbox}
\usepackage{verbatimbox}
\usepackage{fitch}
\usepackage{enumitem}
%\usepackage[defaultsans]{droidsans}
%\usepackage[defaultmono]{droidmono}
%\makeatletter
%\input droid/t1fdm.fd
%\makeatother

\makeatletter
\DeclareRobustCommand*{\bfseries}{%
    \not@math@alphabet\bfseries\mathbf
    \fontseries\bfdefault\selectfont
    \boldmath
}
\makeatother

\makeatletter
\chardef\straightquote@code=\catcode`'
\chardef\backquote@code=\catcode``
\catcode`'=\active \catcode``=\active
\patchcmd{\@noligs}
{\textasciigrave}
{\fixedtextasciigrave}
{}{}
\newcommand{\fixedtextasciigrave}{%
    \makebox[.5em]{\fontencoding{TS1}\fontfamily{fvs}\selectfont\textasciigrave}% Vera Sans
}
\catcode\lq\'=\straightquote@code
\catcode\lq\`=\backquote@code
\makeatletter

\setbeamercovered{again covered={\opaqueness<1->{25}}}

\forestset{close with=\ensuremath{\bigtimes}}

\newcommand{\customframefont}[1]{
    \setbeamertemplate{itemize/enumerate body begin}{#1}
    \setbeamertemplate{itemize/enumerate subbody begin}{#1}
}

\NewEnviron{framefont}[1]{
    \customframefont{#1} % for itemize/enumerate
    {#1 % For the text outside itemize/enumerate
        \BODY
    }
    \customframefont{\normalsize}
}


\makeatletter
\DeclareRobustCommand{\rvdots}{%
    \vphantom{\int\limits^x}\smash[t]{\vdots}
}
\makeatother

\newcommand{\N}{\mathbb{N}}
\newcommand{\Z}{\mathbb{Z}}
\newcommand{\Q}{\mathbb{Q}}
\newcommand{\R}{\mathbb{R}}
\newcommand{\pow}[1]{\mathcal{P}({#1})}
\newcommand{\continuum}{\mathfrak{c}}

\author{Алексей Романов}
\subtitle{Математическая логика и теория алгоритмов}
%\logo{}
\institute{МИЭТ}
\subject{Математическая логика и теория алгоритмов}
%\setbeamercovered{transparent}
%\setbeamertemplate{navigation symbols}{}


\title{Натуральная дедукция (естественный вывод)}
\date{\today}

\begin{document}
\begin{frame}[plain]
    \maketitle
\end{frame}

\begin{frame}
    \frametitle{Секвенции}
    \begin{itemize}
        \item Выражение $A_1,\ldots,A_n \vdash B$ называется секвенцией и читается \enquote{из $A_1,\ldots,A_n$ выводится $B$}.
        \item Удобно считать, что слева множество формул (порядок и повторения неважны).
        \item Оно может быть пустым. \pause
        \item $A_1,\ldots,A_n \vdash B$ истинна, если \pause на всех тех наборах, на которых $A_1,\ldots,A_n$ истинны, \pause $B$ тоже истинна.
        \item Соответственно, $\vdash B$ (\enquote{$B$ выводится}) истинна тогда, когда $B$ \pause тождественно истинна.
    \end{itemize}
\end{frame}

\begin{frame}
    \frametitle{Правила натуральной дедукции}
    \tiny
    \begin{itemize}
        \item Для $\land$:
        \[ 
        \frac{A \qquad B}{A \wedge B}\ \wedge I 
        \qquad \qquad 
        \frac{A \wedge B}{A}\ \wedge{E}
        \qquad
        \frac{A \wedge B}{B}\ \wedge{E} 
        \]
        \item Для $\to$:
        \[ 
        \frac{\boxed{\begin{matrix}
                A \\
                \rvdots \\
                B
        \end{matrix}}}{A \to B}\ \to I 
        \qquad \qquad 
        \frac{A \to B \quad A}{B}\ \to E
        \]
        \item Для $\neg$ и $\bot$:
        \[
        \frac{\boxed{\begin{matrix}
                A \\
                \rvdots \\
                \bot
        \end{matrix}}}{\neg A}\ \neg I 
        \qquad \qquad 
        \frac{\neg A \quad A}{\bot}\ \neg E/\bot I
        \qquad \qquad 
        \frac{\bot}{A}\ \bot{E}
        \]
        \item Для $\lor$:
        \[
        \frac{A}{A \lor B}\ \lor{I}
        \qquad
        \frac{B}{A \lor B}\ \lor{I}
        \qquad \qquad 
        \frac{
            \begin{matrix}
                \phantom{A} \\
                \phantom{\rvdots} \\
                A \lor B 
            \end{matrix}
            \quad
            \boxed{\begin{matrix}
                A \\
                \rvdots \\
                C
            \end{matrix}}
            \quad
            \boxed{\begin{matrix}
                B \\
                \rvdots \\
                C
            \end{matrix}}
        }{C}\ \lor E
        \qquad \qquad         
        \frac{\boxed{\begin{matrix}
                \neg A \\
                \rvdots \\
                B
        \end{matrix}}}{A \lor B}\ \lor I'
        \qquad 
        \frac{\boxed{\begin{matrix}
                \neg B \\
                \rvdots \\
                A
        \end{matrix}}}{A \lor B}\ \lor I'
        \]
        \item Остальные:
        \[
        \frac{\boxed{\begin{matrix}
                \neg A \\
                \rvdots \\
                \bot
        \end{matrix}}}{A}\ RAA
        \qquad \qquad 
        \frac{A}{A}\ R
        \]
        \item $I$ обозначает правила введения (Introduction), $E$ "--- правила исключения (Elimination). 
        \item[] Например, $\land I$ это правило \enquote{введения конъюнкции}. Можете также писать как В$\land$.
    \end{itemize}
\end{frame}

\begin{frame}
    \frametitle{Пример}
    \begin{align*}
        \begin{nd}
            \hypo{1}{(p \land q) \land r} \by{Дано}{}
            \have{2}{p \land q} \ae{1}
            \have[n][-4]{5}{q} \ae{2}
            \have[n][-3]{3}{r} \ae{1}
            \have[n][-2]{4}{p} \ae{2}
            \have[n][-1]{6}{q \land r} \ai{4,5}
            \have{n}{p \land (q \land r)} \ai{3,6}
        \end{nd}
    \end{align*}
\end{frame}

\begin{frame}
    \frametitle{Пример}
    \begin{align*}
        \begin{nd}
            \open
            \onslide<2->    \hypo{prem}{p} \by{Дано}{}
                \open
            \onslide<3->        \hypo{hyp}{q} \by{Дано}{}
            \onslide<3->    \have{reit}{p} \r{prem}
                \close
            \onslide<2->    \have[n][-1]{cond}{q \rightarrow p} \ii{hyp-reit}
            \close
            \onslide<1-> \have[n]{conc}{p \rightarrow (q \rightarrow p)} \ii{prem-cond}
        \end{nd}
    \end{align*}
\end{frame}

\begin{frame}
    \frametitle{Допустимые правила}
    \begin{itemize}
        \item Выше доказали $(p \land q) \land r \vdash p \land (q \land r)$.
        \item Это можно превратить в доказательство $(A \land B) \land C \vdash A \land (B \land C)$ для любых формул $A,B,C$. Как? \pause
        \item Это даёт новое \emph{допустимое правило}:
        \[
            \frac{(A \land B) \land C}{A \land (B \land C)}\ \wedge A
        \]
        \item И так для каждой секвенции, которую докажем! Например
        \[
            \frac{\neg \neg A}{A}\ \neg \neg E \qquad \qquad \frac{A}{\neg \neg A}\ \neg \neg I \qquad \qquad \frac{A \to B \qquad B \to C}{A \to C}\ \to T
        \]
        и т.д.
        \item У каких-то из этих правил есть обозначения, как выше, но можно просто пометить соотвествующей секвенцией.
    \end{itemize}
\end{frame}

\end{document}