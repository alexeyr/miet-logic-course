\documentclass[10pt]{beamer}
\usepackage[utf8]{inputenc}
\usepackage[T1,T2A]{fontenc}
\usepackage[russian]{babel}
\usepackage{color}
\usepackage{calc}
\usepackage{graphicx}
\usepackage{epstopdf}
\usepackage{hyperref}
\hypersetup{unicode,colorlinks}
\usepackage{csquotes}
\usepackage{upquote}
\usepackage{cprotect}
\usetheme[progressbar=head,numbering=fraction,block=fill]{metropolis}
\usepackage{dejavu}
\usepackage{etoolbox}
\usepackage{bm}
\usepackage[ND]{prftree}
\usepackage[tableaux]{prooftrees}
\usepackage{mathtools} % for bigtimes
\usepackage{diagbox}
\usepackage{verbatimbox}
\usepackage{fitch}
\usepackage{enumitem}
%\usepackage[defaultsans]{droidsans}
%\usepackage[defaultmono]{droidmono}
%\makeatletter
%\input droid/t1fdm.fd
%\makeatother

\makeatletter
\DeclareRobustCommand*{\bfseries}{%
    \not@math@alphabet\bfseries\mathbf
    \fontseries\bfdefault\selectfont
    \boldmath
}
\makeatother

\makeatletter
\chardef\straightquote@code=\catcode`'
\chardef\backquote@code=\catcode``
\catcode`'=\active \catcode``=\active
\patchcmd{\@noligs}
{\textasciigrave}
{\fixedtextasciigrave}
{}{}
\newcommand{\fixedtextasciigrave}{%
    \makebox[.5em]{\fontencoding{TS1}\fontfamily{fvs}\selectfont\textasciigrave}% Vera Sans
}
\catcode\lq\'=\straightquote@code
\catcode\lq\`=\backquote@code
\makeatletter

\setbeamercovered{again covered={\opaqueness<1->{25}}}

\forestset{close with=\ensuremath{\bigtimes}}

\newcommand{\customframefont}[1]{
    \setbeamertemplate{itemize/enumerate body begin}{#1}
    \setbeamertemplate{itemize/enumerate subbody begin}{#1}
}

\NewEnviron{framefont}[1]{
    \customframefont{#1} % for itemize/enumerate
    {#1 % For the text outside itemize/enumerate
        \BODY
    }
    \customframefont{\normalsize}
}


\makeatletter
\DeclareRobustCommand{\rvdots}{%
    \vphantom{\int\limits^x}\smash[t]{\vdots}
}
\makeatother

\newcommand{\N}{\mathbb{N}}
\newcommand{\Z}{\mathbb{Z}}
\newcommand{\Q}{\mathbb{Q}}
\newcommand{\R}{\mathbb{R}}
\newcommand{\pow}[1]{\mathcal{P}({#1})}
\newcommand{\continuum}{\mathfrak{c}}

\author{Алексей Романов}
\subtitle{Математическая логика и теория алгоритмов}
%\logo{}
\institute{МИЭТ}
\subject{Математическая логика и теория алгоритмов}
%\setbeamercovered{transparent}
%\setbeamertemplate{navigation symbols}{}


\title{Введение в логику высказываний}
\date{\today}

\begin{document}
\begin{frame}[plain]
\maketitle
\end{frame}

\begin{frame}
\frametitle{Предмет математической логики}
\begin{itemize}
    \item Математическая логика — раздел математики, изучающий математические обозначения, формальные системы, доказуемость математических суждений, природу математического доказательства в целом, вычислимость и прочие аспекты оснований математики (из Wikipedia).
    \item Разделы нашего курса:
    \begin{itemize}
        \item Логика высказываний
        \item Логика предикатов
        \item Теория множеств
        \item Теория алгоритмов (если успеем)
    \end{itemize}
    \item Курс очень короткий и каждый раздел будет на базовом уровне.
    \pause
    \item Сегодня "--- логика высказываний (далее ЛВ). 
\end{itemize}
\end{frame}

\begin{frame}
    \frametitle{Высказывания}
    \begin{itemize}
        \item Высказывания "--- предложения, которые могут быть истинны или ложны.
        \item Например: $4<5$, \enquote{Волга впадает в Балтийское море}.
        \item Но не вопросы, императивные предложения и т.д.
        \pause
        \item Оба эти высказывания \emph{простые} или \emph{атомарные}, так как не содержат более простых.
        \item Их внутренняя структура в ЛВ не рассматривается.
        \pause
        \item Мы можем комбинировать простые высказывания с помощью \emph{логических связок} $\land$ (\enquote{и}, конъюнкция), $\lor$ (\enquote{или}, дизъюнкция), $\to$ (\enquote{если$\ldots$то$\ldots$}, импликация), $\neg$ (\enquote{не}, отрицание).
        \pause
        \item Например, \enquote{в огороде бузина, а в Киеве дядька} = \enquote{в огороде бузина} $\land$ \enquote{в Киеве дядька}.
    \end{itemize}
\end{frame}

\begin{frame}
    \frametitle{Язык логики высказываний}
    \begin{itemize}
        \item Язык ЛВ "--- это \emph{формальный язык}, то есть множество последовательностей символов определённого алфавита, построенных по определённым правилам.
        \pause
        \item Алфавит ЛВ состоит из:
        \begin{itemize}
            \item \emph{Пропозициональных переменных} $p, q, r, p_1,\ldots$ Множество всех переменных обозначим $Var_{Prop}$.
            \item Символов $\land, \lor, \to, \neg$ (иногда включают также $\leftrightarrow$).
            \item Скобок $($, $)$. 
        \end{itemize}
        \pause
        \item Формулы ЛВ (множество $Prop$):
        \begin{itemize}
            \item Каждая переменная "--- формула ($Var_{Prop} \subset Prop$).
            \item Если $A$ "--- формула, то $\neg A$ тоже ($\forall A : Prop ~ \neg A \in Prop$).
            \item Если $A$ и $B$ "--- формулы, то $(A \land B)$, $(A \lor B)$ и $(A \to B)$ тоже ($\forall A, B : Prop ~ (A \land B) \in Prop$, $\ldots$).
            \item Других формул нет.
        \end{itemize}
        \pause
        \item Заметьте, что $A$ и $B$ не принадлежат языку ЛВ. Это \emph{метапеременные}, т.е. переменные \emph{метаязыка}, на котором мы описываем ЛВ. 
    \end{itemize}
\end{frame}

\begin{frame}
    \frametitle{Доказательства свойств формул по индукции}
    \begin{itemize}
        \item Данное определение формул ЛВ "--- \emph{индуктивное} (какие ещё индуктивные определения вы знаете?).
        \item Соответственно, если мы хотим доказать, что какое-то свойство $P$ выполняется для всех формул ($\forall A : Prop ~ P(A)$), достаточно показать, что:
        \pause
        \begin{itemize}
            \item оно верно для всех переменных, $\forall v : Var_{Prop} ~ P(v)$ \\($v$ это снова метапеременная, не переменная ЛВ).
            \item если $A$ "--- формула и $P(A)$ верно, то верно и $P(\neg A)$, $\forall A : Prop ~ (P(A) \to P(\neg A))$.
            \item Если $A$ и $B$ "--- формулы и $P(A)$ и $P(B)$ верны, то $P(A \land B)$, $P(A \lor B)$ и $P(A \to B)$ верны.
        \end{itemize}
        \pause
        \item Пример: покажите, что число открывающих скобок в любой формуле равно числу закрывающих.
        \pause
        \item Более важный: теорема о единственности прочтения. По формуле можно однозначно определить, какая операция была последней в её построении и к каким формулам она была применена.
    \end{itemize}
\end{frame}

\begin{frame}
    \frametitle{Соглашения об опускании скобок}
    \begin{itemize}
        \item TODO
    \end{itemize}
\end{frame}

\begin{frame}
    \frametitle{Семантика логики высказываний}
    \begin{itemize}
        \item Пока мы говорили только о \emph{синтаксисе} языка ЛВ, т.е. какие комбинации символов в нём допустимы.
        \item \emph{Семантика} или \emph{интерпретация} формального языка описывает соответствие между объектами языка и тем, что они обозначают.
        \item У одного языка может быть много интерпретаций. Даже у такого простого, как язык ЛВ. Но мы посмотрим только на стандартную.
        \item Напомним, что высказывания могут быть истинны (обозначим как 1) или ложны (0). $\mathbb{B} = \{0, 1\}$ "--- множество значений истинности.
        \item \emph{Оценка} $\sigma$ это функция $Var_{Prop} \to \mathbb{B}$. Каждую оценку можно продолжить на всё $Prop$ рекурсивно:
        \begin{itemize}
            \item $\sigma(\neg A) = \neg \sigma(A)$
            \item $\ldots$
        \end{itemize}
        \item Заметьте, что $\neg$/$\land$/$\ldots$ слева "--- символы языка ЛВ, а справа "--- операции булевой алгебры.
    \end{itemize}
\end{frame}

\begin{frame}
    \frametitle{Свойства формул}
    \begin{itemize}
        \item Теперь среди всех формул $A$ можно выделить
        \begin{itemize}
            \item \emph{тождественно истинные} (обозначаем как $\vDash A$): $\forall \sigma ~ \sigma(A) = 1$;
            \item \emph{тождественно ложные}: $\forall \sigma ~ \sigma(A) = 0$;
            \item \emph{выполнимые}: $\exists \sigma ~ \sigma(A) = 1$;
            \item \emph{опровержимые}: $\exists \sigma ~ \sigma(A) = 0$.
        \end{itemize}
        \pause
        \item Формулы $A$ и $B$ \emph{эквивалентны} (обозначаем $A \equiv B$), если $\forall \sigma ~ \sigma(A) = \sigma(B)$.
    \end{itemize}
\end{frame}

\begin{frame}
\frametitle{Пример дерева истинности}
\begin{itemize}
    \item Начнём с примера. Нужно доказать $p \to (q \to p \land q)$:
    \item[] 
\only<1>{
    \begin{tableau}{
    }
        [{p \to (q \to p \land q) = 0}, just=Дано]
    \end{tableau}
}
\only<2>{
    \begin{tableau}{
        }
        [{p \to (q \to p \land q) = 0}, just=Дано, checked
        [{p = 1}, just=1
        [{q \to p \land q = 0}, just = 1
        ]]]
    \end{tableau}
}
\only<3>{
    \begin{tableau}{
        }
        [{p \to (q \to p \land q) = 0}, just=Дано, checked
        [{p = 1}, just=1, checked
        [{q \to p \land q = 0}, just = 1
        ]]]
    \end{tableau}
}
\only<4>{
    \begin{tableau}{
        }
        [{p \to (q \to p \land q) = 0}, just=Дано, checked
        [{p = 1}, just=1, checked
        [{q \to p \land q = 0}, just = 1, checked
        [{q = 1}, checked, just=3
        [{p \land q = 0}, just=3
        ]]]]]
    \end{tableau}
}
\only<5>{
    \begin{tableau}{
        }
        [{p \to (q \to p \land q) = 0}, just=Дано, checked
        [{p = 1}, just=1, checked
        [{q \to p \land q = 0}, just = 1, checked
        [{q = 1}, checked, just=3
        [{p \land q = 0}, just=3
        [{p = 0}, just=5] [{q = 0}, just=5]
        ]]]]]
    \end{tableau}
}
\only<6>{
    \begin{tableau}{
        }
        [{p \to (q \to p \land q) = 0}, just=Дано, checked
        [{p = 1}, just=1, checked
        [{q \to p \land q = 0}, just = 1, checked
        [{q = 1}, checked, just=3
        [{p \land q = 0}, just=3
        [{p = 0}, just=5, close={2,6}] [{q = 0}, just=5, close={4,6}]
        ]]]]]
    \end{tableau}
}
\onslide<7->{
    \item Пусть $p \to (q \to p \land q) = 0$. 
\item Тогда $p = 1$ и $q \to p \land q = 0$.
\item Тогда $q = 1$ и $p \land q = 0$.
\item Отсюда $p = 0$ или $q = 0$. Но оба невозможны!
\item Значит,  $p \to (q \to p \land q) = 0$ невозможно.
\item $p \to (q \to p \land q)$ тождественно истинно!
}

\end{itemize}
\end{frame}

\begin{frame}
\frametitle{Правила деревьев истинности}
\begin{itemize}
    \item  Формулы со знаком ($A=0$ или $A=1$) делятся на 4 типа: 
    \begin{itemize}
        \item Атомарные (слева переменная).
        \item Слева отрицание.
        \item $\alpha$ ведут себя как \enquote{и}: $\alpha \Leftrightarrow \alpha_1 \land \alpha_2$.
        \item $\beta$ ведут себя как \enquote{или}: $\beta \Leftrightarrow \beta_1 \lor \beta_2$.
    \end{itemize}
    \pause
    \item[]
    \item Правила:
\begin{columns}
    \begin{column}{0.33\textwidth}
        \begin{center}
            \begin{tableau}{not line numbering, single branches}
                [{\neg A = 1(0)} [{~A = 0(1)}]]
            \end{tableau}
        \end{center}
    \end{column}
    \begin{column}{0.33\textwidth}
        \begin{center}
            \begin{tableau}{not line numbering, single branches}
                [{\alpha} [{\displaystyle \alpha_1 \atop \displaystyle \alpha_2}]]
            \end{tableau}
        \end{center}
    \end{column}
    \begin{column}{0.33\textwidth}
        \begin{center}
            \begin{tableau}{not line numbering}
                [\beta [\beta_1] [\beta_2]]
            \end{tableau}
        \end{center}
    \end{column}
\end{columns}
    \vspace{1em}
    \begin{tabular}{|c | c | c || c | c | c |}
        \hline
        $\alpha$ & $\alpha_1$ & $\alpha_2$ & $\beta$ & $\beta_1$ & $\beta_2$ \\
        \hline
        $A \land B = 1$ & $A = 1$ & $B = 1$ & $A \land B = 0$ & $A = 0$ & $B = 0$ \\
        $A \lor B = 0$ & $A = 0$ & $B = 0$ & $A \lor B = 1$ & $A = 1$ & $B = 1$ \\
        $A \to B = 0$ & $A = 1$ & $B = 0$ & $A \to B = 1$ & $A = 0$ & $B = 1$ \\
        \hline
    \end{tabular}
    \pause
\end{itemize}
\end{frame}

\begin{frame}
\frametitle{Построение дерева истинности}
\begin{itemize}
    \item Идея в поиске контрпримера для формулы (или секвенции), которую хотим доказать.
    \item Для доказательства формулы $A$ мы начинаем с уравнения $A=0$.
    \item Для секвенции $A_1,\, A_2,\,\ldots \vdash B$ начинаем с \pause $A_1=1,\, A_2=1,\, \ldots,\, B=0$ (пишем их в столбик).
    \item Для эквивалентности $A \equiv B$ строим 2 дерева: \pause $A=1,\, B=0$ и $A=0,\, B=1$.
    \pause
    \item[]
    \item На каждом шаге:
    \begin{itemize}
        \item Выбираем неразобранное уравнение (не отмечено \checkmark).
        \item Применяем правила с предыдущего слайда.
        \item Результаты дописываются в конец всех открытых ветвей под ним.
        \item Отмечаем как разобранное.
    \end{itemize}
\end{itemize}
\end{frame}

\begin{frame}
\frametitle{Построение дерева истинности (2)}
\begin{itemize}
    \item Ветвь, в которой одновременно есть $A=1$ и $A=0$ для какой-то формулы $A$, называется \emph{закрытой}.
    \item В ней можно дальше ничего не писать.
    \item[]
    \item Дерево \emph{закрыто}, если все его ветви закрыты. 
    \item В этом случае уравнения, с которых начали, не имеют решений! 
    \item Значит, исходная формула/секвенция не имеет контрпримера и она тождественно истинна.
    \item[]
    \pause
    \item Если какая-то ветвь закончилась (нет неразобранных строк), но не закрылась, то мы нашли контрпример к исходной формуле и она \emph{не} тождественно истинна.
    \item[]
    \pause
    \item Порядок применения влияет только на размер дерева, но не на результат.
    \item Выгоднее сначала применять правила без ветвления.
\end{itemize}
\end{frame}

\begin{frame}
\frametitle{Пример незакрытого дерева}
\begin{itemize}
    \item Проверим $(p \to r) \lor (q \to r) \to (p \lor q \to r)$:
    \begin{tableau}{}
        [{(p \to r) \lor (q \to r) \to (p \lor q \to r) = 0}, checked, just=Дано
        [{(p \to r) \lor (q \to r) = 1}, checked, just=1
        [{p \lor q \to r = 0}, checked, just=1
        [{p \lor q = 1}, checked, just=3
        [{r = 0}, checked, just=3
        [{p \to r = 1}, checked, just=2
        [{p = 1}, checked, just=4
        [{p=0}, checked, just=6, close={7,8}]
        [{r=1}, checked, just=6, close={5,8}]
        ]
        [{q = 1}, checked, just=4
        [\vdots]
        ]
        ]
        [{q \to r = 1}, checked, just=2
        [{p = 1}, checked, just=4
        [{q=0}, checked, just=6]
        [{r=1}, checked, just=6, close={5,8}]
        ]
        [{q = 1}, checked, just=4
        [\vdots]
        ]
        ]
        ]]]]]
    \end{tableau}
    \item В законченной ветви $p=1,\,q=0,\,r=0$. На этом наборе строка 1 выполняется, значит, \pause формула не тождественно истинна.
\end{itemize}
\end{frame}

\begin{frame}
    \frametitle{Корректность и полнота}
    \begin{itemize}
        \item Формальная система \emph{корректна} для какого-то свойства, если все выводимые в этой системе формулы (или другие объекты) имеют это свойство.
        \item Формальная система \emph{полна} для какого-то свойства, если все объекты с этим свойством выводимы.
        \pause
        \item[]
        \item В приложении к деревьям истинности:
        \item Теорема о корректности деревьев истинности для логики высказываний: если для $A=0$ есть закрытое дерево истинности, то $A$ тождественно истинна.
        \item Теорема о полноте деревьев истинности для логики высказываний: если $A$ тождественно истинна, то для $A=0$ есть закрытое дерево истинности.
        \item Можно доказать больше: любое законченное дерево истинности для $A=0$ закрыто.
    \end{itemize}
\end{frame}

\begin{frame}
    \frametitle{Доказательство теоремы о корректности}
    \begin{itemize}
        \item Теорема: если для $A=0$ есть закрытое дерево истинности, то $A$ тождественно истинна.
        \item Доказательство: от противного. Если $A$ не тождественно истинна, то $A=0$ выполнимо.
        \pause
        \item Значит, в исходном дереве единственная ветвь выполнима (есть оценка, для которой все уравнения в этой ветви истинны одновременно).
        \pause
        \item Лемма: если в дереве есть выполнимая ветвь, после применения правил такая ветвь тоже есть. Доказательство отдельно для правил $\neg$, $\alpha$ и $\beta$ (используя $\alpha \Leftrightarrow \alpha_1 \land \alpha_2$ и $\beta \Leftrightarrow \beta_1 \lor \beta_2$).
        \pause
        \item Значит, сколько не применяем правила к исходному дереву, на каждом шаге есть выполнимая ветвь.
        \pause
        \item Закрытая ветвь не может быть выполнима, значит, на каждом шаге есть открытая ветвь.
        \item Значит, дерево не может быть закрыто.
    \end{itemize}
\end{frame}

\begin{frame}
    \frametitle{Доказательство теоремы о полноте}
    \begin{itemize}
        \item Теорема: если $A$ тождественно истинна, то любое законченное дерево истинности для $A=0$ закрыто.
        \item Доказательство: снова от противного. 
        \pause
        \item Пусть есть дерево с законченной открытой ветвью $\mathcal{B}$.
        \pause
        \item Рассмотрим оценку 
        \[ \sigma(v) = \left\{ \begin{array}{l}
            1, \text{ если } v=1 \in \mathcal{B} \\
            0, \text{ если } v=0 \in \mathcal{B} \\
            0 \text{ иначе} 
        \end{array} \right. \]
        \item Лемма: все уравнения в $\mathcal{B}$ выполняются на оценке $\sigma$.
        \pause
        \item Доказательство по индукции: для переменных по определению $\sigma$, и для случаев $\neg$, $\alpha$ и $\beta$.
        \item Для $\beta$: $\beta \in \mathcal{B} \Rightarrow \beta_1 \in \mathcal{B} \lor \beta_2 \in \mathcal{B}$ (т.к. $\beta$ разобрана в $\mathcal{B}$) $\Rightarrow \sigma(\beta_1) = 1 \lor \sigma(\beta_2) = 1$ (по предположению индукции) $\Rightarrow \sigma(\beta) = 1$.
        \pause
        \item Для $\alpha$ аналогично, но $\land$ вместо $\lor$.
        \pause
        \item В частности, $\sigma(A)=0$ и $A$ не тождественно истинна.
    \end{itemize}
\end{frame}

\end{document}